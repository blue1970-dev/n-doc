\newcommand{\thispp}{vpnpp}
\newcommand{\toeversion}{2.11}
\newcommand{\thisTOE}{\thisproduct~\toeversion}
\newcommand{\bsicertid}{BSI-DSZ-CC-xyz}
\newcommand{\bsippid}{BSI-CC-PP-00zz}

% Makro für feste CC-Begriffe wie Enforcing, Required
% Wird nicht vom Autor direkt verwendet, s. dazu
% die folgenden Makros, die direkt den Text setzen.
\newcommand{\ccAttribute}[1]{{\small #1}}
% Makro für "Enforcing"
\newcommand{\enfc}{\secitemfont{SFR-enforcing}}
% Makro für "Supporting"
\newcommand{\supp}{\secitemfont{SFR-supporting}}
% Makro für "non-Interfering"
\newcommand{\noninterfering}{\secitemfont{non-interfering}}
% Makro für "non-TSF"
\newcommand{\nontsf}{\secitemfont{non-TSF}}

% Makros für die Auflösung und Darstellung von Security
% Objectives. Der übergebene Parameter wird anhand der Daten in
% obj.csv aufgelöst.
\newcommand{\obj}[1]{\secitemformat{\directlua{getObjective("#1")}}}
\newcommand{\objtext}[1]{\directlua{getObjectiveText("#1")}}
\newcommand{\objplain}[1]{\directlua{getObjective("#1")}}
\newcommand{\objlink}[1]{\hyperlink{\directlua{toLower("#1")}}{\obj{#1}}}

% Makros für die Auflösung und Darstellung von Security
% Problems. Der übergebene Parameter wird anhand der Daten in
% spd.csv aufgelöst.
\newcommand{\spd}[1]{\secitemformat{\directlua{getSpd("#1")}}}
\newcommand{\spdtext}[1]{\directlua{getSpdText("#1")}}
\newcommand{\spdsource}[1]{\directlua{getSpdSource("#1")}}
\newcommand{\spdlink}[1]{\hyperlink{\directlua{toLower("#1")}}{\spd{#1}}}

%% ------------------------------------------------------------
%%
%% Makros für
%% Security Functional Requirements (SFR)
%%

% Makro für die Auflösung und Darstellung von Security Functional
% Requirements (SFR). Der übergebene Parameter wird anhand der Daten
% in sfr_definitions.tex aufgelöst.
% Das optionale Argument für \sfr{}: Ergänzung (1), (2) z.B. für
% FDP_ACC, FDP_ACF mit vielen Unterpunkten.
\newcommand{\sfr}[2][]{\formatsfr{\directlua{getSfr("#2")}#1}}
\newcommand{\sfrplain}[2][]{\directlua{getSfr("#2")}#1}
\newcommand{\sfrtext}[1]{\directlua{getSfrText("#1")}}
\newcommand{\sfrlink}[2][]{\hyperlink{\directlua{removeSfrSubComponent("#2")}}{\sfr[#1]{#2}}}
% Formatierung der SFR. Kann in seltenen Fällen direkt aufgerufen
% werden (z.B. bei Komponenten-Definitionen, die nicht direkt SFR
% sind).
\newcommand{\formatsfr}[1]{{\secitemformat{#1}}}
% Makro für umgesetzte SFR, kann z.B. am Ende einer Ablaufbeschreibung 
% verwendet werden.
\newcommand{\umgesetztesfr}[1]{\enfsupsfrtable{Umgesetzte SFR}{#1}}
\newcommand{\unterstuetztesfr}[1]{\supportedsfr{#1}}
\newcommand{\supportedsfr}[1]{\enfsupsfrtable{Unterstützte SFR}{#1}}

\newcommand{\enfsupsfrtable}[2]{
  \begin{flushright}
    \minibox[frame,l,t]{{\small \textsf{#1}}\\
      \begin{tabular}{@{}lll@{}}
        #2
      \end{tabular}}
  \end{flushright}}

\newcommand{\getModuleStatus}[1]{\directlua{get_module_status("#1")}}

\newcommand{\checkSfrReferences}{\directlua{checkSfrReferences()}}

%% ------------------------------------------------------------
%%
%% Makros für
%% TOE Security Function Interfaces (TSFI)
%%

% Makros für die Namen der Schnittstellen, die in OPB2.1 eine Rolle
% spielen. 
\newcommand{\ls}{\tsfilink{ls}}

% Formatierung der TSFI wie LS.VPN, LS.LAN etc.
\newcommand{\formatintf}[1]{{\ccfont #1}}
\newcommand{\intf}[1]{\secitemformat{#1}}
\newcommand{\tsfi}[1]{\intf{\directlua{getTsfi("#1")}}}
\newcommand{\tsfilink}[1]{\hyperlink{tsfi.#1}{\tsfi{#1}}}
\newcommand{\printSfToTsfi}[1]{\directlua{print_sf_to_tsfi_table("#1")}}
\newcommand{\printTsfiToSf}[1]{\directlua{print_tsfi_to_sf_table("#1")}}

%% ------------------------------------------------------------
%%
%% Makros für
%% TOE Security Functions (SF)
%%

% Makros für die Sicherheitsfunktionen. Der übergebene Parameter wird
% anhand der Daten in sf_definitions.tex aufgelöst.
\newcommand{\secfunc}[1]{\formatsecfunc{\directlua{getSecfunc("#1")}}}
\newcommand{\secfuncplain}[1]{\directlua{getSecfunc("#1")}}
\newcommand{\secfunctext}[1]{\directlua{getSecfuncText("#1")}}
\newcommand{\secfunclink}[1]{\hyperlink{\directlua{toLower("#1")}}{\secfunc{#1}}}
\newcommand{\formatsecfunc}[1]{{\secitemformat{#1}}}
\newcommand{\secfuncheadline}[1]{\texorpdfstring{\secfunctext{#1}\hspace{1em}(\secitemfont{\secfuncplain{#1}})}{\secfunctext{#1} (\secfuncplain{#1})}}

% Makro für den Verweis auf Subsysteme, Module und Schnittstellen, die im
% TDS definifiert werden.
\newcommand{\tds}[2][no]{\secitemformat{\directlua{replacelabel("#2", "#1")}}}
\newcommand{\tdsplain}[2][no]{\directlua{replacelabel("#2", "#1")}}
% Makro für Links auf Subsysteme/Module. Ist nur für das TDS relevant.
% Wird im TDS so umdefiniert, dass die Link-Funktion dazukommt.
% Für alle anderen Dokumente wird stattdessen das \tds Makto aufgerufen.
\newcommand{\tdslink}[2][no]{\tds[#1]{#2}}





%%% Local Variables:
%%% mode: latex
%%% TeX-master: shared
%%% TeX-engine: luatex
%%% End:
