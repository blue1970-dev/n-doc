% Allgemeine Makros für die Titelseite
% Können nach einem ersten Release dokumentspezifisch
% überschrieben werden.
\newcommand{\documentdate}[1]{\directlua{getDocumentDate("#1")}}
\newcommand{\documentversion}[1]{\directlua{getDocumentVersion("#1")}}

% Name des Produkts in Kurz- und in Langform
\newcommand{\thisproduct}{MauveVPN Client}
\newcommand{\thisproductlong}{MauveCorp MauveVPN Client}
\newcommand{\thisdeveloper}{Mauve Corporation}
\newcommand{\thiscompany}{MauveCorp\\Fliederweg 98\\50020 Köln\\\textsmaller{certification@mauvecorp.com}}

\newcommand{\thispp}{vpnpp}
\newcommand{\toeversion}{2.11}
\newcommand{\thisTOE}{\thisproduct~\toeversion}
\newcommand{\bsicertid}{BSI-DSZ-CC-xyz}
\newcommand{\bsippid}{BSI-CC-PP-00zz}




% Als Erinnerung für später zu erledigende Aufgaben
% Markiert den übergebenen Text deutlich und setzt
% einen Merker in die Randspalte.
% define list of To Do items
\DeclareNewTOC[listname={Offene Punkte}]{todo}
\newcommand{\todocolor}{magenta}
\newcommand{\todo}[2][]{\marginpar{\textcolor{\todocolor}%
    {\fbox{\textbf{To Do}}}}\textcolor{\todocolor}{#2}
  \ifthenelse{\equal{#1}{}}{}{\addcontentsline{todo}{section}{#1}}}


% Kann verwendet werden, wenn in eigenen Dokumenten Abschnitte leer
% bleiben, aber die Kongruenz zum PP oder zu anderen CC Dokumenten
% erhalten bleiben soll.
\newcommand{\intentionallyleftblank}{(This section intentionally left blank.)}

% Makro für Java (und anderen) Quelltext
\newcommand{\code}[1]{\texttt{#1}}
\newcommand{\java}[2][]{\textsmaller[2]{\code{#1}}\code{#2}}
\newcommand{\bundle}[1]{\footnotesize{\java{#1}}}

% Makro für feste CC-Begriffe wie Enforcing, Required
% Wird nicht vom Autor direkt verwendet, s. dazu
% die folgenden Makros, die direkt den Text setzen.
\newcommand{\ccAttribute}[1]{{\small #1}}
% Makro für "Enforcing"
\newcommand{\enfc}{\secitemfont{SFR-enforcing}}
% Makro für "Supporting"
\newcommand{\supp}{\secitemfont{SFR-supporting}}
% Makro für "non-Interfering"
\newcommand{\noninterfering}{\secitemfont{non-interfering}}
% Makro für "non-TSF"
\newcommand{\nontsf}{\secitemfont{non-TSF}}

% Makro für RFC. Argument ist die Nummer des RFC. Enthält das optionale
% Argument den Buchstaben "c", wird auch eine Referenz auf das
% Literaturverzeichnis eingefügt.
% Achtung: Das Leerzeichen vor \autocite ist wichtig.
\newcommand{\rfc}[2][false]{RFC\,#2%
  \ifthenelse{\equal{#1}{c}}{ \autocite{rfc#2}}{}}

\newcommand{\nounderscore}[1]{\directlua{replaceUnderscore("#1")}}

% Makro für Schlüsselwörter, z.B. Konfigurationsparameter aus der
% Admin-Anwendung.  Vieles, was man ansonsten in Anführungszeichen
% schreiben würde, kann besser als \keyword{} ausgezeichnet werden.
% Die Unterstriche werden automatisch mit einem Backslash maskiert.
\newcommand{\keyword}[1]{\textsmaller[0.8]{\ccfont \textit{\directlua{replaceUnderscore("#1")}}}}

\newcommand{\errorcondition}[1]{{\keyword{#1}}}

% Makro für Fehlercode mit Fehlertext
\newcommand{\error}[1]{#1 -- \enquote{\directlua{getError("#1")}}}

% Makro für Dateinamen
\newcommand{\filename}[1]{{\narrowfilefont #1}}

% Links auf Abbildungen, Tabellen, Abschnitte
\newcommand{\figureref}[1]{\hyperref[#1]{Abbildung~\ref*{#1}}}
\newcommand{\tableref}[1]{\hyperref[#1]{Tabelle~\ref*{#1}}}
\newcommand{\listref}[1]{\hyperref[#1]{Listing~\ref*{#1}}}
\newcommand{\sectref}[1]{\hyperref[#1]{Abschnitt~\ref*{#1}}}
\newcommand{\chapterref}[1]{\hyperref[#1]{Kapitel~\ref*{#1}}}
\newcommand{\appendixref}[1]{\hyperref[#1]{Anhang~\ref*{#1}}}

% Überschriften, die direkt den Anker für einen Hyperlink enthalten
\newcommand{\hrefchapter}[2]{\hypertarget{#1}{\chapter{#2}\label{#1}}}
\newcommand{\hrefsection}[2]{\hypertarget{#1}{\section{#2}\label{#1}}}
\newcommand{\hrefsubsection}[2]{\hypertarget{#1}{\subsection{#2}\label{#1}}}
\newcommand{\hrefsubsubsection}[2]{\hypertarget{#1}{\subsubsection{#2}\label{#1}}}
\newcommand{\hrefparagraph}[2]{\hypertarget{#1}{\paragraph{#2}\label{#1}}}
\newcommand{\hrefsubparagraph}[2]{\hypertarget{#1}{\subparagraph{#2}\label{#1}}}


% Allgemeines Format für Begriffe aus der CC:
% Objectives (O.)
% Objectives of the environment (OE.)
% Threats (T.)
% Assumptions (A.)
% SFR
% SF
%
\newcommand{\secitemfont}[1]{{\ccfont #1}}
\newcommand{\secitemformat}[1]{{\textsmaller[0.9]{\secitemfont{#1}}}}
\newcommand{\secitem}[1]{\secitemformat{\nounderscore{#1}}}


% Makros für die Auflösung und Darstellung von Security
% Objectives. Der übergebene Parameter wird anhand der Daten in
% obj.csv aufgelöst.
\newcommand{\obj}[1]{\secitemformat{\directlua{getObjective("#1")}}}
\newcommand{\objtext}[1]{\directlua{getObjectiveText("#1")}}
\newcommand{\objplain}[1]{\directlua{getObjective("#1")}}
\newcommand{\objlink}[1]{\hyperlink{\directlua{toLower("#1")}}{\obj{#1}}}

% Makros für die Auflösung und Darstellung von Security
% Problems. Der übergebene Parameter wird anhand der Daten in
% spd.csv aufgelöst.
\newcommand{\spd}[1]{\secitemformat{\directlua{getSpd("#1")}}}
\newcommand{\spdtext}[1]{\directlua{getSpdText("#1")}}
\newcommand{\spdsource}[1]{\directlua{getSpdSource("#1")}}
\newcommand{\spdlink}[1]{\hyperlink{\directlua{toLower("#1")}}{\spd{#1}}}

%% ------------------------------------------------------------
%%
%% Makros für
%% Security Functional Requirements (SFR)
%%

% Makro für die Auflösung und Darstellung von Security Functional
% Requirements (SFR). Der übergebene Parameter wird anhand der Daten
% in sfr_definitions.tex aufgelöst.
% Das optionale Argument für \sfr{}: Ergänzung (1), (2) z.B. für
% FDP_ACC, FDP_ACF mit vielen Unterpunkten.
\newcommand{\sfr}[2][]{\formatsfr{\directlua{getSfr("#2")}#1}}
\newcommand{\sfrplain}[2][]{\directlua{getSfr("#2")}#1}
\newcommand{\sfrtext}[1]{\directlua{getSfrText("#1")}}
\newcommand{\sfrlink}[2][]{\hyperlink{\directlua{removeSfrSubComponent("#2")}}{\sfr[#1]{#2}}}
% Formatierung der SFR. Kann in seltenen Fällen direkt aufgerufen
% werden (z.B. bei Komponenten-Definitionen, die nicht direkt SFR
% sind).
\newcommand{\formatsfr}[1]{{\secitemformat{#1}}}
% Makro für umgesetzte SFR, kann z.B. am Ende einer Ablaufbeschreibung 
% verwendet werden.
\newcommand{\umgesetztesfr}[1]{\enfsupsfrtable{Umgesetzte SFR}{#1}}
\newcommand{\unterstuetztesfr}[1]{\supportedsfr{#1}}
\newcommand{\supportedsfr}[1]{\enfsupsfrtable{Unterstützte SFR}{#1}}

\newcommand{\enfsupsfrtable}[2]{
  \begin{flushright}
    \minibox[frame,l,t]{{\small \textsf{#1}}\\
      \begin{tabular}{@{}lll@{}}
        #2
      \end{tabular}}
  \end{flushright}}

\newcommand{\getModuleStatus}[1]{\directlua{get_module_status("#1")}}

\newcommand{\checkSfrReferences}{\directlua{checkSfrReferences()}}


%% ------------------------------------------------------------
%%
%% Makros für
%% TOE Security Function Interfaces (TSFI)
%%

% Makros für die Namen der Schnittstellen, die in OPB2.1 eine Rolle
% spielen. 
\newcommand{\ls}{\tsfilink{ls}}

% Formatierung der TSFI wie LS.VPN, LS.LAN etc.
\newcommand{\formatintf}[1]{{\ccfont #1}}
\newcommand{\intf}[1]{\secitemformat{#1}}
\newcommand{\tsfi}[1]{\intf{\directlua{getTsfi("#1")}}}
\newcommand{\tsfilink}[1]{\hyperlink{tsfi.#1}{\tsfi{#1}}}
\newcommand{\printSfToTsfi}[1]{\directlua{print_sf_to_tsfi_table("#1")}}
\newcommand{\printTsfiToSf}[1]{\directlua{print_tsfi_to_sf_table("#1")}}

\newcommand{\lslan}{\tsfilink{ls.lan}}
\newcommand{\lswan}{\tsfilink{ls.wan}}
\newcommand{\lswandhcp}{\tsfilink{ls.wan.dhcp}}
\newcommand{\lsled}{\tsfilink{ls.led}}
\newcommand{\lslantls}{\tsfilink{ls.lan.tls}}
\newcommand{\lslanhttpmgmt}{\tsfilink{ls.lan.httpmgmt}}

%% ------------------------------------------------------------
%%
%% Makros für
%% TOE Security Functions (SF)
%%

% Makros für die Sicherheitsfunktionen. Der übergebene Parameter wird
% anhand der Daten in sf_definitions.tex aufgelöst.
\newcommand{\secfunc}[1]{\formatsecfunc{\directlua{getSecfunc("#1")}}}
\newcommand{\secfuncplain}[1]{\directlua{getSecfunc("#1")}}
\newcommand{\secfunctext}[1]{\directlua{getSecfuncText("#1")}}
\newcommand{\secfunclink}[1]{\hyperlink{\directlua{toLower("#1")}}{\secfunc{#1}}}
\newcommand{\formatsecfunc}[1]{{\secitemformat{#1}}}
\newcommand{\secfuncheadline}[1]{\texorpdfstring{\secfunctext{#1}\hspace{1em}(\secitemfont{\secfuncplain{#1}})}{\secfunctext{#1} (\secfuncplain{#1})}}

%% ------------------------------------------------------------
%%
%% Makros für
%% Tabellen
%% 
\newcommand{\tcheck}{\textsmaller[1]{\checkmark{}}}
\newcommand{\tno}{\ldotp{}}
\newcommand{\toptio}{\textcolor{blue}{\textbullet{}}}
\newcommand{\tdrop}{\textcolor{blue}{\textopenbullet{}}}
\newcommand{\tadded}{\textcolor{magenta}{\tcheck{}}}
\newcommand{\leftblank}{---}

%% ------------------------------------------------------------
%%
%% Makros für
%% IDs der TLS Verbindungen
%%
%% Aufbau der Tabelle
\newcounter{tlsconnid}
\renewcommand*\thetlsconnid{TLS.\arabic{tlsconnid}}
\newcommand{\tlsconntablerow}[2]{\luadirect{getTlsConnectionTableRow("#1", "#2")}}
\newcommand{\tlsid}[1]{\refstepcounter{tlsconnid}\secitemformat{\thetlsconnid}\label{#1}}

%% Makros für die Referenzierung einer TLS ID
\newcommand{\tlsexample}{\intf{\ref{tls.example}}}

% Makro für den Verweis auf Subsysteme, Module und Schnittstellen, die im
% TDS definifiert werden.
\newcommand{\tds}[2][no]{\secitemformat{\directlua{replacelabel("#2", "#1")}}}
\newcommand{\tdsplain}[2][no]{\directlua{replacelabel("#2", "#1")}}
% Makro für Links auf Subsysteme/Module. Ist nur für das TDS relevant.
% Wird im TDS so umdefiniert, dass die Link-Funktion dazukommt.
% Für alle anderen Dokumente wird stattdessen das \tds Makto aufgerufen.
\newcommand{\tdslink}[2][no]{\tds[#1]{#2}}

% Drei Spaltentypen für Tabellen, mehrzeiligen Text mit erzwungenen Zeilenumbrüchen
% erlauben. Vgl. https://tex.stackexchange.com/questions/12703/
\newcolumntype{L}[1]{>{\raggedright\let\newline\\\arraybackslash\hspace{0pt}}p{#1}}
\newcolumntype{C}[1]{>{\centering\let\newline\\\arraybackslash\hspace{0pt}}p{#1}}
\newcolumntype{R}[1]{>{\raggedleft\let\newline\\\arraybackslash\hspace{0pt}}p{#1}}
\newcolumntype{Y}{>{\raggedright\let\newline\\\arraybackslash\hspace{0pt}}X}


\newcommand{\landscapetable}[1]{%
\afterpage{%
  \clearpage% Flush earlier floats (otherwise order might not be correct)
  \begin{landscape}% Landscape page
    \centering % Center table
    {
      \input{#1}
    }
  \end{landscape}
  \clearpage% Flush page
}%
}

% Auslassungspunkte zwischen erster und zweiter Spalte einer Tabelle
\makeatletter
\newcommand \mydotfill {\leavevmode \cleaders \hb@xt@ .66em{\hss .\hss }\hfill \kern \z@}
\makeatother
\newcommand{\tablekeyword}[1]{\keyword{#1}\mydotfill}
\newcommand{\defaultvalue}[1]{\textsl{#1}}

% Kommentare für Verzeichnisbäume, z.B. im FSP
\newcommand{\myDTcomment}[1]{\mydotfill{}\begin{minipage}[t]{10cm}\textrm{#1}\end{minipage}}

\newcolumntype{Q}[2]{%
    >{\adjustbox{angle=#1,lap=\width-(#2)}\bgroup}%
    l%
    <{\egroup}%
}
\newcommand*\rot{\multicolumn{1}{Q{90}{0em}}}% no optional argument here, please!


%%% Local Variables:
%%% mode: latex
%%% TeX-master: shared
%%% TeX-engine: luatex
%%% End:
