% Allgemeine Makros für die Titelseite
% Können nach einem ersten Release dokumentspezifisch
% überschrieben werden.
\newcommand{\documentdate}[1]{\directlua{getDocumentDate("#1")}}
\newcommand{\documentversion}[1]{\directlua{getDocumentVersion("#1")}}
\newcommand{\nosnapshot}[1]{\directlua{tex.sprint(get_version_number_for_refliste("#1"))}}

% Als Erinnerung für später zu erledigende Aufgaben
% Markiert den übergebenen Text deutlich und setzt
% einen Merker in die Randspalte.
% define list of To Do items
\DeclareNewTOC[listname={Offene Punkte}]{todo}
\newcommand{\todocolor}{magenta}
\newcommand{\todo}[2][]{\marginpar{\textcolor{\todocolor}%
    {\fbox{\textbf{To Do}}}}\textcolor{\todocolor}{#2}
  \ifthenelse{\equal{#1}{}}{}{\addcontentsline{todo}{section}{#1}}}


% Kann verwendet werden, wenn in eigenen Dokumenten Abschnitte leer
% bleiben, aber die Kongruenz zum PP oder zu anderen CC Dokumenten
% erhalten bleiben soll.
\newcommand{\intentionallyleftblank}{(This section intentionally left blank.)}

% Makro für Java (und anderen) Quelltext
\newcommand{\code}[1]{\texttt{#1}}
\newcommand{\java}[2][]{\textsmaller[2]{\code{#1}}\code{#2}}
\newcommand{\bundle}[1]{\footnotesize{\java{#1}}}

% Makro für RFC. Argument ist die Nummer des RFC. Enthält das optionale
% Argument den Buchstaben "c", wird auch eine Referenz auf das
% Literaturverzeichnis eingefügt.
% Achtung: Das Leerzeichen vor \autocite ist wichtig.
\newcommand{\rfc}[2][false]{RFC\,#2%
  \ifthenelse{\equal{#1}{c}}{ \autocite{rfc#2}}{}}

\newcommand{\nounderscore}[1]{\directlua{replaceUnderscore("#1")}}

% Makro für Schlüsselwörter, z.B. Konfigurationsparameter aus der
% Admin-Anwendung.  Vieles, was man ansonsten in Anführungszeichen
% schreiben würde, kann besser als \keyword{} ausgezeichnet werden.
% Die Unterstriche werden automatisch mit einem Backslash maskiert.
\newcommand{\keyword}[1]{\textsmaller[0.8]{\ccfont \textit{\directlua{replaceUnderscore("#1")}}}}

\newcommand{\errorcondition}[1]{{\keyword{#1}}}

% Makro für Fehlercode mit Fehlertext
\newcommand{\error}[1]{#1 -- \enquote{\directlua{getError("#1")}}}

% Makro für Dateinamen
\newcommand{\filename}[1]{{\narrowfilefont #1}}

% Links auf Abbildungen, Tabellen, Abschnitte
\newcommand{\figureref}[1]{\hyperref[#1]{Abbildung~\ref*{#1}}}
\newcommand{\tableref}[1]{\hyperref[#1]{Tabelle~\ref*{#1}}}
\newcommand{\listref}[1]{\hyperref[#1]{Listing~\ref*{#1}}}
\newcommand{\sectref}[1]{\hyperref[#1]{Abschnitt~\ref*{#1}}}
\newcommand{\chapterref}[1]{\hyperref[#1]{Kapitel~\ref*{#1}}}
\newcommand{\appendixref}[1]{\hyperref[#1]{Anhang~\ref*{#1}}}

% Überschriften, die direkt den Anker für einen Hyperlink enthalten
\newcommand{\hrefchapter}[2]{\hypertarget{#1}{\chapter{#2}\label{#1}}}
\newcommand{\hrefsection}[2]{\hypertarget{#1}{\section{#2}\label{#1}}}
\newcommand{\hrefsubsection}[2]{\hypertarget{#1}{\subsection{#2}\label{#1}}}
\newcommand{\hrefsubsubsection}[2]{\hypertarget{#1}{\subsubsection{#2}\label{#1}}}
\newcommand{\hrefparagraph}[2]{\hypertarget{#1}{\paragraph{#2}\label{#1}}}
\newcommand{\hrefsubparagraph}[2]{\hypertarget{#1}{\subparagraph{#2}\label{#1}}}


% Allgemeines Format für Begriffe aus der CC:
% Objectives (O.)
% Objectives of the environment (OE.)
% Threats (T.)
% Assumptions (A.)
% SFR
% SF
%
\newcommand{\secitemfont}[1]{{\ccfont #1}}
\newcommand{\secitemformat}[1]{{\textsmaller[0.9]{\secitemfont{#1}}}}
\newcommand{\secitem}[1]{\secitemformat{\nounderscore{#1}}}

%% ------------------------------------------------------------
%%
%% Makros für
%% Tabellen
%% 
\newcommand{\tcheck}{\textsmaller[1]{\checkmark{}}}
\newcommand{\tno}{\ldotp{}}
\newcommand{\toptio}{\textcolor{blue}{\textbullet{}}}
\newcommand{\tdrop}{\textcolor{blue}{\textopenbullet{}}}
\newcommand{\tadded}{\textcolor{magenta}{\tcheck{}}}
\newcommand{\leftblank}{---}

%% ------------------------------------------------------------
%%
%% Makros für
%% IDs der TLS Verbindungen
%%
%% Aufbau der Tabelle
\newcounter{tlsconnid}
\renewcommand*\thetlsconnid{TLS.\arabic{tlsconnid}}
\newcommand{\tlsconntablerow}[2]{\luadirect{getTlsConnectionTableRow("#1", "#2")}}
\newcommand{\tlsid}[1]{\refstepcounter{tlsconnid}\secitemformat{\thetlsconnid}\label{#1}}

% Drei Spaltentypen für Tabellen, mehrzeiligen Text mit erzwungenen Zeilenumbrüchen
% erlauben. Vgl. https://tex.stackexchange.com/questions/12703/
\newcolumntype{L}[1]{>{\raggedright\let\newline\\\arraybackslash\hspace{0pt}}p{#1}}
\newcolumntype{C}[1]{>{\centering\let\newline\\\arraybackslash\hspace{0pt}}p{#1}}
\newcolumntype{R}[1]{>{\raggedleft\let\newline\\\arraybackslash\hspace{0pt}}p{#1}}
\newcolumntype{Y}{>{\raggedright\let\newline\\\arraybackslash\hspace{0pt}}X}
\newcolumntype{Q}[2]{>{\adjustbox{angle=#1,lap=\width-(#2)}\bgroup}l<{\egroup}}
\newcommand*\rot{\multicolumn{1}{Q{90}{0em}}}% no optional argument here, please!

\newcommand{\landscapetable}[1]{%
\afterpage{%
  \clearpage% Flush earlier floats (otherwise order might not be correct)
  \begin{landscape}% Landscape page
    \centering % Center table
    {
      \input{#1}
    }
  \end{landscape}
  \clearpage% Flush page
}%
}

% Auslassungspunkte zwischen erster und zweiter Spalte einer Tabelle
\makeatletter
\newcommand \mydotfill {\leavevmode \cleaders \hb@xt@ .66em{\hss .\hss }\hfill \kern \z@}
\makeatother
\newcommand{\tablekeyword}[1]{\keyword{#1}\mydotfill}
\newcommand{\defaultvalue}[1]{\textsl{#1}}

%%% Local Variables:
%%% mode: latex
%%% TeX-master: shared
%%% TeX-engine: luatex
%%% End:
