%%%%%%%%%%%%%%%%%%%%%%%%%%%%%%%%%%%%%%%%%%%%%%%%%%%%%%%%%%%%%%%%%%%%%%%%%%%%
%
% Macros for authoring the TOE Design Specification
% =================================================
%
% This file defines macros that are required for authoring the TDS modules.  
%
%%%%%%%%%%%%%%%%%%%%%%%%%%%%%%%%%%%%%%%%%%%%%%%%%%%%%%%%%%%%%%%%%%%%%%%%%%%%


% Macro for "Provided"
\newcommand{\provided}{\secitemfont{(Provided)}}

% Macro for "Required"
\newcommand{\required}{\secitemfont{(Required)}}

%% Headlines for the subsystem section The description of the subsystem should
%% use exactly thse macros for headlines. They provide the correct level of
%% outline and create hypertext anchors. Argument is the key of the subsystem as
%% defined in the database in /common/db/subsystems.csv

%% Note about the implementation: Direct Lua calls must be used here, no other macros. Texts are used as PDF bookmarks and macros are not resolved in PDF bookmarks. 
\newcommand{\subsystemchapter}[1]{\clearpage\hrefsection{#1}{Subsystem \texorpdfstring{\secitemfont{\tdsplain{#1}}}{\directlua{replacelabelplain("#1", "no")}}}}
\newcommand{\subsystemdescription}[1]{\subsection*{Description}\label{#1}}
\newcommand{\subsystemsfr}[1]{\subsection*{Implemented SFR}\label{{#1}}}
\newcommand{\subsysteminteract}[1]{\subsection*{Interaction With Other Subsystems}\label{#1}}
\newcommand{\modulesforsubsystemchapter}[2]{\clearpage\section{\texorpdfstring{Modules for Subsystem \secitemfont{\directlua{replacelabel("#2", "no")}}}{Modules for Subsystem \directlua{replacelabelplain("#2", "no")}}}\label{#1}}

%% Headlines for the module section The description of the module should
%% use exactly thse macros for headlines. They provide the correct level of
%% outline and create hypertext anchors. Argument is the key of the module as
%% defined in the database in /common/db/modules.csv
%%
\newcommand{\modulechapter}[1]{\hrefsubsection{#1}{Module \texorpdfstring{\secitemfont{\directlua{replacelabel("#1", "fq")}}}{\directlua{replacelabelplain("#1", "x")}}}}
  
\newcommand{\moduledescription}[1]{\hrefsubsubsection{#1}{Description}}
\newcommand{\moduleprocesses}[1]{\hrefsubsubsection{#1}{Processes}}
\newcommand{\moduleinterfaces}[1]{\hrefsubsubsection{#1}{Interfaces To Other Modules}}

%% Fest formulierte Überschriften for die Modulbeschreibungen
%% Argument 1: Das Label des Hypertextankers
%% Argument 2: Der Text der Überschrift
%%
\newcommand{\modulAblauf}[2]{\hrefparagraph{#1}{#2}}
\newcommand{\modulAblaufSchritt}[2]{\hrefsubparagraph{#1}{#2}}

%% Fest formulierte Überschriften for die Beschreibung der
%% Schnittstellen eines Moduls.
%% Parameter: Label der Schnittstelle
%% Das Label muss in der Regel noch in die tdsDefinitions eingetragen werden.
\newcommand{\modulSchnittstelleProvided}[1]{\hrefparagraph{#1}{\tdsplain{#1}\,\provided{}}}
\newcommand{\modulSchnittstelleRequired}[1]{\paragraph{\tdsplain{#1}\,\required{}}}

\newcommand{\generatesfrtable}[1]{%
  \noindent{}Das Modul erfüllt die Anforderungen, die durch die SFR in
  \tableref{tab:#1.sfr} an den EVG gestellt werden. Das Modul ist
  \getModuleStatus{#1} for den TOE.\par{}
  \captionsetup[table]{list=no}
  \begin{table}[htb]
    \centering
    \printModuleToSfrTable[enf]{#1}
    \printModuleToSfrTable[sup]{#1}
    \caption{SFR des Moduls \tds[fq]{#1}}
    \label{tab:#1.sfr}
  \end{table}
  \captionsetup[table]{list=yes}
}

\newcommand{\generateSfrToModulesTable}{
  \begin{longtable}[c]{@{}ll@{\hskip 1cm}l@{}}
    \toprule
    \secitem{SFR} &  Relation &  \secitem{Subsystem::Modul} \\ \endfirsthead
    
    \toprule \secitem{SFR} &  Relation &  \secitem{Subsystem::Modul} \\ \midrule \endhead
    \multicolumn{3}{r}{\rule{0pt}{3ex}\emph{Weiter auf der nächsten Seite}} \endfoot
    \endlastfoot
    \printModulesForSfrRows{}
    \bottomrule
    \caption{Zuordnung von Modulen zu SFRs}
    \label{tab:sfr2modules}
  \end{longtable}
}

\newcommand{\generatesfrsubsystable}[1]{Das Subsystem erfüllt die
  Anforderungen, die durch die SFR in \tableref{tab:#1.sfr.enf} und \tableref{tab:#1.sfr.sup} an den EVG
  gestellt werden.\par
  \captionsetup[table]{list=no}
  \begin{table}[p]
    \centering
    \printSubsystemToSfrTable[enf]{#1}
    \caption{Enforcing SFR des Subsystems \tds[fq]{#1}}
    \label{tab:#1.sfr.enf}
  \end{table}
  \begin{table}[bp]
    \centering
    \printSubsystemToSfrTable[sup]{#1}{}
    \caption{Supporting SFR des Subsystems \tds[fq]{#1}}
    \label{tab:#1.sfr.sup}
  \end{table}
  \captionsetup[table]{list=yes}
}

\newcommand{\generatebundletable}[1]{Das Modul setzt sich aus den Bundles zusammen, die in \tableref{tab:#1.bundles} aufgezählt sind.
  \captionsetup[table]{list=no}
  \begin{table}[htb]
    \centering
    \printModuleToBundleTable{#1}
    \caption{Dekomposition des Moduls \tds[fq]{#1} in Bundles}
    \label{tab:#1.bundles}
  \end{table}
  \captionsetup[table]{list=yes}
}



% !TEX root = adv_tds
%%% Local Variables:
%%% mode: latex
%%% TeX-master: "adv_tds"
%%% TeX-engine: luatex
%%% End:
