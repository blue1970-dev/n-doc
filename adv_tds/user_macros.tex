%%%%%%%%%%%%%%%%%%%%%%%%%%%%%%%%%%%%%%%%%%%%%%%%%%%%%%%%%%%
%
% Makros für die Autoren der TDS Module
% =====================================
%
% Hier werden Makros definiert, die für die Beschreibung
% der TDS Module notwendig sind. Diese Liste stellt
% gewissermaßen die API dar. Bitte keine direkten
% Formatierungen mit LaTeX Befehlen benutzen, sondern
% schauen was
% a) in dieser Datei
% b) in /common/common-macros.tex
% definiert ist. Im Zweifel bitte Rücksprache mit AKR
% halten.
%
%%%%%%%%%%%%%%%%%%%%%%%%%%%%%%%%%%%%%%%%%%%%%%%%%%%%%%%%%%%



% Makro für "Provided"
\newcommand{\provided}{\ccAttribute{(Provided)}}

% Makro für "Required"
\newcommand{\required}{\ccAttribute{(Required)}}

% Standardisiertes Einbinden der Grafik eines Subsystems
% Argument 1: Label der Grafik für Referenzen auf das Bild
% Argument 2: Pfad zur Grafikdatei
% Argument 3: Name des dargestellten Subsystems
\newcommand{\subsystemdecompositionfigure}[3]{%
\ccFigure{#1}{subsysteme/#2}{Dekomposition des Subsystems #3 in Module}{0.5}}

% Standardisiertes Einbinden der Grafik eines Moduls
% Argument 1: Label der Grafik für Referenzen auf das Bild
% Argument 2: Pfad zur Grafikdatei
% Argument 3: Name des dargestellten Subsystems
\newcommand{\moduledecompositionfigure}[3]{%
\ccFigure{#1}{#2}{Dekomposition des Moduls #3}{1.0}}

%% Fest formulierte Überschriften für das Kapitel Subsysteme.
%% Die Beschreibung jedes Subsystems muss genau diese Überschriften enthalten.
%% zu übergebendes Argument: Das Label (der Hypertextanker) dieses Abschnitts.
%%
%% Notiz zur Implementierung: Hier müssen direkte Lua-Aufrufe stehen, kein
%% \tdsplain{} Makro, da sonst die unaufgelösten Strings in die PDF-Bookmarks
%% übernommen werden.
\newcommand{\subsystemkapitel}[1]{\clearpage\hrefsection{#1}{Subsystem \texorpdfstring{\secitemfont{\tdsplain{#1}}}{\directlua{replacelabelplain("#1", "no")}}}}
\newcommand{\subsystembeschreibung}[1]{\subsection*{Beschreibung}\label{#1}}
\newcommand{\subsystemsfr}[1]{\subsection*{Umgesetzte SFR}\label{{#1}}}
\newcommand{\subsysteminteract}[1]{\subsection*{Interaktion mit anderen Subsystemen}\label{#1}}
\newcommand{\modulefuersubsystemkapitel}[2]{\clearpage\section{\texorpdfstring{Module für Subsystem \secitemfont{\directlua{replacelabel("#2", "no")}}}{Module für Subsystem \directlua{replacelabelplain("#2", "no")}}}\label{#1}}

%% Fest formulierte Überschriften für das Kapitel Module.
%% Die Beschreibung jedes Moduls muss genau diese Überschriften enthalten.
%% zu übergebendes Argument: Das Label (der Hypertextanker) dieses Abschnitts.
%%
\newcommand{\moduleKapitel}[1]{\hrefsubsection{#1}{Modul \texorpdfstring{\secitemfont{\directlua{replacelabel("#1", "fq")}}}{\directlua{replacelabelplain("#1", "x")}}}}
  
\newcommand{\moduleBeschreibung}[1]{\hrefsubsubsection{#1}{Beschreibung}}
\newcommand{\moduleAblaeufe}[1]{\hrefsubsubsection{#1}{Abläufe des Moduls}}
\newcommand{\moduleSchnittstellen}[1]{\hrefsubsubsection{#1}{Schnittstellen zu anderen Modulen}}

%% Fest formulierte Überschriften für die Modulbeschreibungen
%% Argument 1: Das Label des Hypertextankers
%% Argument 2: Der Text der Überschrift
%%
\newcommand{\modulAblauf}[2]{\hrefparagraph{#1}{#2}}
\newcommand{\modulAblaufSchritt}[2]{\hrefsubparagraph{#1}{#2}}

%% Fest formulierte Überschriften für die Beschreibung der
%% Schnittstellen eines Moduls.
%% Parameter: Label der Schnittstelle
%% Das Label muss in der Regel noch in die tdsDefinitions eingetragen werden.
\newcommand{\modulSchnittstelleProvided}[1]{\hrefparagraph{#1}{\tdsplain{#1}\,\provided{}}}
\newcommand{\modulSchnittstelleRequired}[1]{\paragraph{\tdsplain{#1}\,\required{}}}

\newcommand{\generatesfrtable}[1]{%
  \noindent{}Das Modul erfüllt die Anforderungen, die durch die SFR in
  \tableref{tab:#1.sfr} an den EVG gestellt werden. Das Modul ist
  \modulStatus{#1} für den TOE.\par{}
  \captionsetup[table]{list=no}
  \begin{table}[htb]
    \centering
    \printModuleToSfrTable[enf]{#1}
    \printModuleToSfrTable[sup]{#1}
    \caption{SFR des Moduls \tds[fq]{#1}}
    \label{tab:#1.sfr}
  \end{table}
  \captionsetup[table]{list=yes}
}

\newcommand{\generateSfrToModulesTable}{
  \begin{longtable}[c]{@{}ll@{\hskip 1cm}l@{}}
    \toprule
    \secitem{SFR} &  Relation &  \secitem{Subsystem::Modul} \\ \endfirsthead
    
    \toprule \secitem{SFR} &  Relation &  \secitem{Subsystem::Modul} \\ \midrule \endhead
    \multicolumn{3}{r}{\rule{0pt}{3ex}\emph{Weiter auf der nächsten Seite}} \endfoot
    \endlastfoot
    \printModulesForSfrRows{}
    \bottomrule
    \caption{Zuordnung von Modulen zu SFRs}
    \label{tab:sfr2modules}
  \end{longtable}
}

\newcommand{\generatesfrsubsystable}[1]{Das Subsystem erfüllt die
  Anforderungen, die durch die SFR in \tableref{tab:#1.sfr.enf} und \tableref{tab:#1.sfr.sup} an den EVG
  gestellt werden.\par
  \captionsetup[table]{list=no}
  \begin{table}[p]
    \centering
    \printSubsystemToSfrTable[enf]{#1}
    \caption{Enforcing SFR des Subsystems \tds[fq]{#1}}
    \label{tab:#1.sfr.enf}
  \end{table}
  \begin{table}[bp]
    \centering
    \printSubsystemToSfrTable[sup]{#1}{}
    \caption{Supporting SFR des Subsystems \tds[fq]{#1}}
    \label{tab:#1.sfr.sup}
  \end{table}
  \captionsetup[table]{list=yes}
}

\newcommand{\generatebundletable}[1]{Das Modul setzt sich aus den Bundles zusammen, die in \tableref{tab:#1.bundles} aufgezählt sind.
  \captionsetup[table]{list=no}
  \begin{table}[htb]
    \centering
    \printModuleToBundleTable{#1}
    \caption{Dekomposition des Moduls \tds[fq]{#1} in Bundles}
    \label{tab:#1.bundles}
  \end{table}
  \captionsetup[table]{list=yes}
}



% !TEX root = adv_tds
%%% Local Variables:
%%% mode: latex
%%% TeX-master: "adv_tds"
%%% TeX-engine: luatex
%%% End:
