%% Beispieldatei für die Beschreibung eines TDS Moduls.
%% Das wichtigste an dieser Stelle ist, genau das Kürzel des Moduls
%% zu kennen, das beschrieben wird. Module heißen immer
%%
%% mod.<subsystem>.<modul>
%%
%% wobei die entsprechenden Definitionen in der Datei
%% modules.csv stehen.
%%
%% Kleiner Tipp: Diese Datei hat als Vorlage
%%
%% mod.subxxx.modyyy
%%
%% Mit einem Suche/Ersetze Lauf zum Austausch des Template-Codes
%% kann man sich schon eine Menge Arbeit sparen.


\moduleKapitel{mod.cryptsystem.algorithms}

%% Generiert die Tabelle der SFR, die dieses Modul umsetzt oder
%% unterstützt.
\generatesfrtable{mod.cryptsystem.algorithms}

%% Generiert die Tabelle der Bundles, aus denen sich dieses Modul
%% zusammensetzt
%\generatebundletable{mod.cryptsystem.algorithms}

\moduleBeschreibung{mod.cryptsystem.algorithms.desc}

Über das Modul \tds{mod.cryptsystem.algorithms} des Subsystems
\tdslink{sub.cryptsystem} stellt der TOE
Verfahren .......

% Hier die Funktionen des Moduls beschreiben: Was macht es? Welche
% Rolle spielt es im Gesamtsystem? Noch nicht in die Details gehen,
% diese werden später bei den Abläufen erklärt.

%% Nach der Beschreibung des Moduls folgt die Beschreibung der
%% Abläufe.  Diese Überschrift leitet den entsprechenden Abschnitt
%% ein.
\moduleAblaeufe{prc.cryptsystem.algorithms}

%% Für jeden Ablauf gibt es eine solche Überschrift. Der letzte Teil
%% sollte nicht zu lang, aber ausreichend verständlich sein.
\modulAblauf{prc.cryptsystem.algorithms.hash}{Hashwerte berechnen}

An dieser Stelle wird die Berechnung des Hashwertes beschrieben.

\umgesetztesfr{\sfrlink{fcs_cop.1/hash}}

\modulAblauf{prc.cryptsystem.algorithms.hmac}{HMAC berechnen}

An dieser Stelle wird die Berechnung des HMAC beschrieben.

\umgesetztesfr{\sfrlink{fcs_cop.1/hmac}}


%% Der nächste wichtige Abschnitt in der Modulbeschreibung: Die
%% Schnittstellen.
\moduleSchnittstellen{int.cryptsystem.algorithms}


%% Jede Schnittstelle hat einen Bezeichner, der mit "int."
%% anfängt. Eine Schnittstelle ist entweder provided (das Modul bietet
%% die Schnittstelle an) oder required (das Modul nutzt die
%% Schnittstelle). Die Provided-Schnittstellen kommen zuerst.
\modulSchnittstelleProvided{int.cryptsystem.algorithms.gethash}

Über diese Schnittstelle werden Hashwerte berechnet (Ablauf in \sectref{prc.cryptsystem.algorithms.hash}).

\modulSchnittstelleProvided{int.cryptsystem.algorithms.gethmac}

Über diese Schnittstelle werden HMAC berechnet (Ablauf in \sectref{prc.cryptsystem.algorithms.hmac}).


% !TEX root = "../../adv_tds"
%%% Local Variables:
%%% mode: latex
%%% TeX-master: "../../adv_tds"
%%% TeX-engine: luatex
%%% End:
