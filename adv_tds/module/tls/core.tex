%% Beispieldatei für die Beschreibung eines TDS Moduls.
%% Das wichtigste an dieser Stelle ist, genau das Kürzel des Moduls
%% zu kennen, das beschrieben wird. Module heißen immer
%%
%% mod.<subsystem>.<modul>
%%
%% wobei die entsprechenden Definitionen in der Datei
%% modules.csv stehen.
%%
%% Kleiner Tipp: Diese Datei hat als Vorlage
%%
%% mod.subxxx.modyyy
%%
%% Mit einem Suche/Ersetze Lauf zum Austausch des Template-Codes
%% kann man sich schon eine Menge Arbeit sparen.


\moduleKapitel{mod.tls.core}

%% Generiert die Tabelle der SFR, die dieses Modul umsetzt oder
%% unterstützt.
\generatesfrtable{mod.tls.core}

%% Generiert die Tabelle der Bundles, aus denen sich dieses Modul
%% zusammensetzt
%\generatebundletable{mod.tls.core}

\moduleBeschreibung{mod.tls.core.desc}

Über das Modul \tds{mod.tls.core} des Subsystems
\tdslink{sub.tls} stellt der TOE
Schnittstellen .......

% Hier die Funktionen des Moduls beschreiben: Was macht es? Welche
% Rolle spielt es im Gesamtsystem? Noch nicht in die Details gehen,
% diese werden später bei den Abläufen erklärt.

%% Nach der Beschreibung des Moduls folgt die Beschreibung der
%% Abläufe.  Diese Überschrift leitet den entsprechenden Abschnitt
%% ein.
\moduleAblaeufe{prc.tls.core}

%% Für jeden Ablauf gibt es eine solche Überschrift. Der letzte Teil
%% sollte nicht zu lang, aber ausreichend verständlich sein.
\modulAblauf{prc.tls.core.connect}{Aufbau einer TLS-Verbindung}

\umgesetztesfr{\sfrlink{ftp_itc.1/tls} & \sfrlink{fcs_cop.1/tls.aes}}

\modulAblauf{prc.tls.core.auth}{Authentisieren des Kommunikationspartners}

\umgesetztesfr{\sfrlink{fcs_cop.1/tls.auth}}

\modulAblauf{prc.tls.core.disconnect}{Abbau einer TLS-Verbindung}

\umgesetztesfr{\sfrlink{ftp_itc.1/tls}}

%% Hier folgt die ausführliche Beschreibung des SFR-relevanten
%% Ablaufs.  Das sollte ordentlich in die Details des Verfahrens
%% gehen. Es ist nicht wichtig, auf Implementationsdetails
%% herumzureiten. Hingegen ist es für den Evaluator hilfreich, wenn
%% bestimmte Namen genannt werden, weil das einen
%% schönen Anker für das Lesen des Codes bildet. Auf keinen Fall
%% sollten hier Zeilennummern genannt werden. Diese müssten jedesmal
%% wieder angepasst und korrigiert werden.
%%
%% Wenn auf Code verwiesen wird, bitte mit dem Makro \code{}
%% einschließen. Ggf. kann es notwendig sein, Trennhilfen
%% vorzugeben. Diese bitte mit "\-" an den entsprechenden Stellen
%% einfügen:  \code{get\-Card\-Terminals()}
%%


%% Wenn an einzelnen Stellen besonders deutlich erkennbar ist, dass
%% ein SFR umgesetzt wird, ist es hilfreich, den Evaluator darauf
%% hinzuweisen. Auf jeden Fall muss eine solche Zeile am Ende einer
%% Ablaufbeschreibung stehen.



%% Der nächste wichtige Abschnitt in der Modulbeschreibung: Die
%% Schnittstellen.
\moduleSchnittstellen{int.tls.core}


%% Jede Schnittstelle hat einen Bezeichner, der mit "int."
%% anfängt. Eine Schnittstelle ist entweder provided (das Modul bietet
%% die Schnittstelle an) oder required (das Modul nutzt die
%% Schnittstelle). Die Provided-Schnittstellen kommen zuerst.
\modulSchnittstelleProvided{int.tls.core.accept}

Über diese Schnittstelle wird eine TLS-Verbindung aufgebaut.

\modulSchnittstelleProvided{int.tls.core.disconnect}

Über diese Schnittstelle wird eine TLS-Verbindung abgebaut.

\modulSchnittstelleRequired{int.tls.cert.checkcert}

Die Schnittstelle \tdslink[fq]{int.tls.cert.checkcert} wird verwendet, um das
Zertifikat des Kommunikationspartners zu prüfen.

% !TEX root = "../../adv_tds"
%%% Local Variables:
%%% mode: latex
%%% TeX-master: "../../adv_tds"
%%% TeX-engine: luatex
%%% End:
