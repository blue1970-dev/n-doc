%% Beispieldatei für die Beschreibung eines TDS Moduls.
%% Das wichtigste an dieser Stelle ist, genau das Kürzel des Moduls
%% zu kennen, das beschrieben wird. Module heißen immer
%%
%% mod.<subsystem>.<modul>
%%
%% wobei die entsprechenden Definitionen in der Datei
%% modules.csv stehen.
%%
%% Kleiner Tipp: Diese Datei hat als Vorlage
%%
%% mod.vpn.core
%%
%% Mit einem Suche/Ersetze Lauf zum Austausch des Template-Codes
%% kann man sich schon eine Menge Arbeit sparen.


\moduleKapitel{mod.vpn.core}

%% Generiert die Tabelle der SFR, die dieses Modul umsetzt oder
%% unterstützt.
\generatesfrtable{mod.vpn.core}

%% Generiert die Tabelle der Bundles, aus denen sich dieses Modul
%% zusammensetzt
%\generatebundletable{mod.vpn.core}

\moduleBeschreibung{mod.vpn.core.desc}

Über das Modul \tds{mod.vpn.core} des Subsystems
\tdslink{sub.vpn} stellt der TOE
Schnittstellen .......

% Hier die Funktionen des Moduls beschreiben: Was macht es? Welche
% Rolle spielt es im Gesamtsystem? Noch nicht in die Details gehen,
% diese werden später bei den Abläufen erklärt.

%% Nach der Beschreibung des Moduls folgt die Beschreibung der
%% Abläufe.  Diese Überschrift leitet den entsprechenden Abschnitt
%% ein.
\moduleAblaeufe{prc.vpn.core}

%% Für jeden Ablauf gibt es eine solche Überschrift. Der letzte Teil
%% sollte nicht zu lang, aber ausreichend verständlich sein.
\modulAblauf{prc.vpn.core.connect}{Verbindung zum VPN-Konzentrator aufbauen}

Der Ablauf baut die Verbindung zum VPN-Konzentrator auf. Das Zertifikat des
Konzentrators wird über die Schnittstelle \tdslink[fq]{int.vpn.cert.checkcert}
geprüft.

\umgesetztesfr{\sfrlink{ftp_itc.1/vpn} & \sfrlink{fcs_ckm.2}}

%% Für jeden Ablauf gibt es eine solche Überschrift. Der letzte Teil
%% sollte nicht zu lang, aber ausreichend verständlich sein.
\modulAblauf{prc.vpn.core.disconnect}{Verbindung zum VPN-Konzentrator abbbauen}

Der Ablauf baut die Verbindung zum VPN-Konzentrator ab.

\umgesetztesfr{\sfrlink{ftp_itc.1/vpn}}


%% Der nächste wichtige Abschnitt in der Modulbeschreibung: Die
%% Schnittstellen.
\moduleSchnittstellen{int.vpn.core}


%% Jede Schnittstelle hat einen Bezeichner, der mit "int."
%% anfängt. Eine Schnittstelle ist entweder provided (das Modul bietet
%% die Schnittstelle an) oder required (das Modul nutzt die
%% Schnittstelle). Die Provided-Schnittstellen kommen zuerst.
\modulSchnittstelleProvided{int.vpn.core.connect}

Über diese Schnittstelle wird die VPN-Verbindung aufgebaut
(vgl. \sectref{prc.vpn.core.connect}).

\modulSchnittstelleProvided{int.vpn.core.disconnect}

Über diese Schnittstelle wird die VPN-Verbindung abgebaut (vgl. \sectref{prc.vpn.core.disconnect}).

%% Ab hier: Die Liste der Schnittstellen, die entweder über die
%% Blueprint-Konfiguration oder programmatisch herangezogen werden.
\modulSchnittstelleRequired{int.vpn.cert.checkcert}

Die Schnittstelle \tdslink[fq]{int.vpn.cert.checkcert} wird
verwendet, um das Zertifikat des VPN-Konzentrators zu prüfen.

\modulSchnittstelleRequired{int.cryptsystem.keymgmt.createkey}

Die Schnittstelle \tdslink[fq]{int.cryptsystem.keymgmt.createkey} wird
verwendet, um einen Verbindungsschlüssel zu erzeugen.

\modulSchnittstelleRequired{int.cryptsystem.keymgmt.destroykey}

Die Schnittstelle \tdslink[fq]{int.cryptsystem.keymgmt.destroykey} wird
verwendet, um einen Verbindungsschlüssel zu zerstören.


% !TEX root = "../../adv_tds"
%%% Local Variables:
%%% mode: latex
%%% TeX-master: "../../adv_tds"
%%% TeX-engine: luatex
%%% End:
