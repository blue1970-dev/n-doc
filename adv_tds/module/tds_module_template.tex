%% Beispieldatei für die Beschreibung eines TDS Moduls.
%% Das wichtigste an dieser Stelle ist, genau das Kürzel des Moduls
%% zu kennen, das beschrieben wird. Module heißen immer
%%
%% mod.<subsystem>.<modul>
%%
%% wobei die entsprechenden Definitionen in der Datei
%% modules.csv stehen.
%%
%% Kleiner Tipp: Diese Datei hat als Vorlage
%%
%% mod.subxxx.modyyy
%%
%% Mit einem Suche/Ersetze Lauf zum Austausch des Template-Codes
%% kann man sich schon eine Menge Arbeit sparen.


\moduleKapitel{mod.subxxx.modyyy}

%% Generiert die Tabelle der SFR, die dieses Modul umsetzt oder
%% unterstützt.
\generatesfrtable[Modul]{mod.subxxx.modyyy}

%% Generiert die Tabelle der Bundles, aus denen sich dieses Modul
%% zusammensetzt
\generatebundletable{mod.subxxx.modyyy}

\moduleBeschreibung{mod.subxxx.modyyy.desc}

Über das Modul \tds{mod.subxxx.modyyy} des Subsystems
\tdslink{sub.subxxx} stellt der TOE
Schnittstellen .......

% Hier die Funktionen des Moduls beschreiben: Was macht es? Welche
% Rolle spielt es im Gesamtsystem? Noch nicht in die Details gehen,
% diese werden später bei den Abläufen erklärt.

%% Nach der Beschreibung des Moduls folgt die Beschreibung der
%% Abläufe.  Diese Überschrift leitet den entsprechenden Abschnitt
%% ein.
\moduleAblaeufe{prc.subxxx.modyyy}

%% Für jeden Ablauf gibt es eine solche Überschrift. Der letzte Teil
%% sollte nicht zu lang, aber ausreichend verständlich sein.
\modulAblauf{prc.subxxx.modyyy.validateUserInput}{Prüfung auf
  Gültigkeit der Benutzereingaben}

%% Hier folgt die ausführliche Beschreibung des SFR-relevanten
%% Ablaufs.  Das sollte ordentlich in die Details des Verfahrens
%% gehen. Es ist nicht wichtig, auf Implementationsdetails
%% herumzureiten. Hingegen ist es für den Evaluator hilfreich, wenn
%% bestimmte Namen genannt werden, weil das einen
%% schönen Anker für das Lesen des Codes bildet. Auf keinen Fall
%% sollten hier Zeilennummern genannt werden. Diese müssten jedesmal
%% wieder angepasst und korrigiert werden.
%%
%% Wenn auf Code verwiesen wird, bitte mit dem Makro \code{}
%% einschließen. Ggf. kann es notwendig sein, Trennhilfen
%% vorzugeben. Diese bitte mit "\-" an den entsprechenden Stellen
%% einfügen:  \code{get\-Card\-Terminals()}
%%


%% Wenn an einzelnen Stellen besonders deutlich erkennbar ist, dass
%% ein SFR umgesetzt wird, ist es hilfreich, den Evaluator darauf
%% hinzuweisen. Auf jeden Fall muss eine solche Zeile am Ende einer
%% Ablaufbeschreibung stehen.
\umgesetztesfr{\sfrlink{fmt_msa.1/ak.infomod} && \sfrlink{fmt_msa.3/ak.infomod}}


%% Der nächste wichtige Abschnitt in der Modulbeschreibung: Die
%% Schnittstellen.
\moduleSchnittstellen{int.subxxx.modyyy}


%% Jede Schnittstelle hat einen Bezeichner, der mit "int."
%% anfängt. Eine Schnittstelle ist entweder provided (das Modul bietet
%% die Schnittstelle an) oder required (das Modul nutzt die
%% Schnittstelle). Die Provided-Schnittstellen kommen zuerst.
\modulSchnittstelleProvided{int.subxxx.modyyy.icfgmgmt}

Über diese Schnittstelle wird die Konfiguration des Moduls
\tds{mod.subxxx.modyyy} verwaltet. Die Konfiguration kann darüber entweder
gelesen oder geschrieben werden.

\modulSchnittstelleProvided{int.subxxx.modyyy.iaas}

Über diese Schnittstelle bietet \tds{mod.subxxx.modyyy} den anderen Modulen
des TOE Dienste für Zugriffsberechtigungen auf dynamische und
statische Aspekte des Informationsmodells an.

%% Ab hier: Die Liste der Schnittstellen, die entweder über die
%% Blueprint-Konfiguration oder programmatisch herangezogen werden.
\modulSchnittstelleRequired{int.cardservice.core.icardservice}

Die Schnittstelle \tdslink[fq]{int.cardservice.core.icardservice} wird
verwendet, um Informationen über die zur Verfügung stehenden Smart
Cards abzurufen.

% !TEX root = "../../adv_tds"
%%% Local Variables:
%%% mode: latex
%%% TeX-master: "../../adv_tds"
%%% TeX-engine: luatex
%%% End:
