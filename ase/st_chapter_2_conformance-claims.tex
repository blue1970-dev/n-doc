\hrefchapter{conf}{Postulat der Übereinstimmung} %{Conformance Claims}

\hrefsection{conf.cc}{Konformität zu Common
  Criteria} %{Common Criteria Conformance Claim}

Das Security Target wurde gemäß Common Criteria, Version~3.1, Revision~5, erstellt und ist

\begin{itemize}
\item CC Part 2 \autocite{CCPart2} erweitert (extended) und
\item CC Part 3 \autocite{CCPart3} konform (conformant).
\end{itemize}

\hrefsection{conf.pp}{Konformität zu Schutzprofilen}

Dieses Security Target behauptet strikte Konformität zu:

\begin{itemize}
\item "`\citetitle{\thispp}"' \autocite{\thispp}
\end{itemize}

\noindent{}Dieses Security Target behauptet keine Konformität zu weiteren Schutzprofilen.

\hrefsection{conf.package}{Konformität zu Paketen} %{Package Conformance Claim}

Das Schutzprofil fordert die Vertrauenswürdigkeitsstufe EAL3, erweitert um die
Komponenten in \tableref{tab:conf.eal3plus}. Dieses Security Target behauptet
Konformität zu genau diesen Paketen. Diese Konformität wird als „EAL3+“
bezeichnet und ist somit „package-augmented“ gegenüber EAL3.

\begin{table}[htbp]
  \centering
  \begin{tabularx}{0.7\textwidth}{@{}lY@{}}
    \toprule
    Paket & Erläuterung \\
    \midrule
    \formatsfr{AVA\_VAN.5} & Resistenz gegen Angriffspotential „Enhanced-Basic“\\
    \formatsfr{ADV\_FSP.4} & Vollständige Funktionale Spezifikation\\
    \formatsfr{ADV\_TDS.3} & Einfaches Modulares Design\\
    \formatsfr{ADV\_IMP.1} & TSF-Implementierung\\
    \formatsfr{ALC\_TAT.1} & Wohldefinierte Entwicklungswerkzeuge\\
    \formatsfr{ALC\_FLR.2} & Verfahren für Problemreports\\
    \bottomrule
  \end{tabularx}
  \caption{Ergänzungen zur Vertrauenswürdigkeit EAL3}
  \label{tab:conf.eal3plus}
\end{table}

\hrefsection{conf.rationale}{Erklärung der Konformität} %{Conformance Rationale}

Dieses Security Target behauptet strikte Konformität zu \autocite{\thispp}. Durch diese
Feststellung sind Widersprüche und Inkonsistenzen zu anderen Schutzprofilen
ausgeschlossen. Diese Behauptung basiert auf der Betrachtung des TOE Typs, der
Definition des Sicherheitsproblems und schließlich der Sicherheitsziele sowie
der Sicherheitsanforderungen. Weiterhin behauptet dieses Security Target
Konformität zu allen Security Assurance Requirements (SARs), die von
\autocite{\thispp} gefordert werden.

\begin{description}

\item[TOE Typ] Das Schutzprofil fordert, dass der TOE ein VPN Router ist. Der
  vorliegende TOE ist ein VPN Router.
  
\item[Definition des Sicherheitsproblems] Die Definition des
  Sicherheitsproblems, d.\,h. die Bedrohungen, Annahmen und die
  organisatorischen Sicherheitspolitiken sind direkt aus dem Schutzprofil
  \citepp{} übernommen.
  
\item[Sicherheitsziele und Sicherheitsanforderungen] Die Sicherheitsziele und
  Sicherheitsanforderungen sind dem Schutzprofil \citepp{}
  entnommen. Die Operationen an den SFR sind deutlich gekennzeichnet.
\end{description}

\chapterref{comp} bescheibt die über CC~Teil~2 \autocite{CCPart2} hinausgehenden
funktionalen Anforderungen an die Vertrauenswürdigkeit. Es werden keine
Anforderungen definiert, die über CC~Teil~3 \autocite{CCPart3} hinausgehen.

%%% Local Variables:
%%% mode: latex
%%% TeX-master: shared
%%% TeX-engine: luatex
%%% End:
