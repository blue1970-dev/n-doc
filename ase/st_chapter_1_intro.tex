\hrefchapter{intro}{Einführung in das Security Target}

Der TOE, der in diesem Dokument beschrieben wird, ist der \emph{\thisTOE{}}. Der
TOE ist eine sichere Komponente, die als \thisproduct{} eingesetzt wird.

Dieses Dokument ist das \emph{Security Target}, in dem die funktionalen und
organisatorischen Sicherheitsanforderungen des TOE und seiner Einsatzumgebung
beschrieben werden.  Dieses Dokument findet seine formale Grundlage in:

\begin{itemize}
\item \citetitle{\thispp} \autocite{\thispp}
\end{itemize}

\hrefsection{intro.refst}{ST Referenz}

\begin{tabularx}{1\textwidth}{@{}p{0.3\textwidth}X@{}}
  \toprule
  Titel des Dokuments & Security Target / \thisproduct\\
  Version des Dokuments & \documentversion{\thisdocument}\\
  Datum des Dokuments & \documentdate{\thisdocument}\\
  Allgemeiner Status: & \\
  Autor  &  \thisdeveloper{}\\
  Editor &  \\
  \bottomrule
\end{tabularx}

\hrefsection{intro.reftoe}{TOE Referenz}

\begin{tabularx}{1\textwidth}{@{}p{0.3\textwidth}X@{}}
  \toprule
  Evaluierungsgegenstand & \thisTOE{}\\
  Version des EVG & \toeversion{} \\
  Hersteller & \thisdeveloper{} \\
  Vertrauenswürdigkeitsstufe &
  \secitemformat{EAL3} erweitert um
  \secitemformat{AVA\_VAN.3}, \secitemformat{ADV\_IMP.1}, %
  \secitemformat{ADV\_TDS.3}, \secitemformat{ADV\_FSP.4}, %
  \secitemformat{ALC\_TAT.1}, and \secitemformat{ALC\_FLR.2} %
  (Kurzbezeichnung „\secitemformat{EAL3+}“)\\
  CC Version &  3.1 Release 5\\
  \bottomrule
\end{tabularx}

\cleardoublepage{}

\hrefsection{intro.overview}{Überblick über den TOE}

Der Evaluierungsgegenstand ist der \thisTOE{}.

Der Lieferumfang des TOE umfasst ebenfalls die Betriebsdokumentation für
\thisproduct{}. Somit entspricht der TOE dem im Schutzprofil \autocite{\thispp}
genannten Umfang und Aufbau.

\hrefsubsection{intro.overview.toetype}{TOE Typ}

\thisproduct{} implementiert -- konform zu \citepp{} -- den Produkttyp eines
VPN-Routers.

\hrefsubsection{intro.overview.usage}{Verwendung und Hauptfunktionalität des
  TOE}

\hrefsubsection{intro.overview.nontoe}{Erforderliche Non-TOE
  Hardware/Software/Firmware}


\hrefsection{intro.desc}{Beschreibung des TOE}

\hrefsubsection{intro.desc.goals}{Hauptziele des TOE}

\hrefsubsection{intro.desc.aufbau}{Aufbau des TOE}

% \begin{figure}[tb]
%   \centering{}
%   \hypertarget{fig:intro.desc.case}{%
%     \includegraphics[width=0.8\columnwidth,keepaspectratio]%
%     {../common/media/thetoe_front_1.jpg}
%     \includegraphics[width=0.8\columnwidth,keepaspectratio]%
%     {../common/media/thetoe_back_1.jpg}
%     \caption{Abbildung des \thisproduct}
%     \label{fig:intro.desc.case}}
% \end{figure}

Das Betriebssystem des \thisproduct{} ist GNU/Linux. Teile des Betriebssystems
setzen Sicherheitsanforderungen an den TOE um und sind somit SFR-enforcing.

\hrefsubsection{intro.desc.env}{Einsatzumgebung des TOE}

\hrefsubsection{intro.desc.hardware}{Hardware des \thisproduct}

\hrefsubsection{intro.desc.intf}{Schnittstellen des \thisproduct}

\hrefsubsubsection{intro.desc.intf.phys}{Physische Schnittstellen}

Alle Schnittstellen des thisproduct{} sind physisch am Gehäuse des Geräts
untergebracht. Die außen sichtbaren Schnittstellen sind auf dem Foto des TOE in
\figureref{fig:intro.desc.case} zu erkennen.

\begin{description}
\item[\intf{PS.LAN}] ist die Schnittstelle ins LAN\dots
\item[\intf{PS.WAN}] ist die Schnittstelle ins WAN\dots
\item[\intf{PS.LED}] repräsentiert die LEDs an der Außenseite des Geräts.
\end{description}

\hrefsubsubsection{intro.desc.intf.log}{Logische Schnittstellen}

Der TOE verfügt über die logischen Schnittstellen, die das Schutzprofil
\autocite{\thispp} beschreibt. Diese werden hier der besseren Lesbarkeit halber
wiederholt.

\begin{description}

\item[\hypertarget{tsfi.ls.lan}{\intf{LS.LAN}}] ist die Schnittstelle des TOE
  ins lokale Netzwerk der Einsatzumgebung. Zusätzlich zu den im Schutzprofil
  genannten Schnittstellen werden hier weitere protokollspezifische
  Schnittstellen definiert. \tableref{tab:tsfi.ls.lan} listet diese logischen
  Schnittstellen.

\item[\hypertarget{tsfi.ls.wan}{\intf{LS.WAN}}] ist die Schnittstelle des TOE
  zum WA. Verschiedene Protokolle implementieren weitere Logische Schnittstellen
  in Richtung des WAN. \tableref{tab:tsfi.ls.wan} listet diese logischen
  Schnittstellen.
 
\item[\hypertarget{tsfi.ls.led}{\intf{LS.LED}}] repräsentiert die
  logische Schnittstelle zum Display und den Bedienknöpfen über
  \intf{PS.LED}.

\end{description}

  \begin{table}[htbp]
    \centering
    \begin{tabularx}{1\columnwidth}{@{}llX@{}}
      \toprule
      Bezeichner & Rolle & Zweck der Schnittstelle \\
      \midrule
      \hypertarget{tsfi.ls.lan.ether}{\tsfi{ls.lan.ether}} & --- & Protokoll auf Zugangsschicht \\
      \hypertarget{tsfi.ls.lan.ip}{\tsfi{ls.lan.ip}} & --- & Zugang zur Internet-Schicht \\
      \hypertarget{tsfi.ls.lan.tcp}{\tsfi{ls.lan.tcp}} & --- & Zugang zur Transportschicht \\
      \hypertarget{tsfi.ls.lan.tls}{\tsfi{ls.lan.tls}} & beide & Sicherung der Verbindung mit TLS~1.2 \\
      \hypertarget{tsfi.ls.lan.udp}{\tsfi{ls.lan.udp}} & --- & Zugang zur Transportschicht \\
      \hypertarget{tsfi.ls.lan.httpmgmt}{\tsfi{ls.lan.httpmgmt}} & Server & HTTP-Zugriff auf Managementschnittstelle \\
      \bottomrule
    \end{tabularx}
    \caption{Logische Schnittstellen an \tsfi{ls.lan}}
    \label{tab:tsfi.ls.lan}
  \end{table}

  \begin{table}[htbp]
    \centering
    \begin{tabularx}{1\columnwidth}{@{}llX@{}}
      \toprule
      Bezeichner & Rolle & Zweck der Schnittstelle \\
      \midrule
      \hypertarget{tsfi.ls.wan.ether}{\tsfi{ls.wan.ether}} & --- & Protokoll auf Zugangsschicht \\
      \hypertarget{tsfi.ls.wan.ip}{\tsfi{ls.wan.ip}} & --- & Zugang zur Internet-Schicht \\
      \hypertarget{tsfi.ls.wan.ntp}{\tsfi{ls.wan.ntp}} & Client & Abruf der Uhrzeit \\
      \hypertarget{tsfi.ls.wan.dhcp}{\tsfi{ls.wan.dhcp}} & Client & Adressbezug im WAN \\
      \hypertarget{tsfi.ls.wan.tcp}{\tsfi{ls.wan.tcp}} & --- & Zugang zur Transportschicht \\
      \hypertarget{tsfi.ls.wan.udp}{\tsfi{ls.wan.udp}} & --- & Zugang zur Transportschicht \\
      \hypertarget{tsfi.ls.wan.ipsec}{\tsfi{ls.wan.ipsec}} & --- & VPN Datenverkehr \\
      \bottomrule
    \end{tabularx}
    \caption{Logische Schnittstellen an \tsfi{ls.wan}}
    \label{tab:tsfi.ls.wan}
  \end{table}


%%% Local Variables:
%%% mode: latex
%%% TeX-master: t
%%% End:
 % Wird auch in anderen Dokumenten verwendet

\clearpage

\hrefsubsection{intro.desc.physdiff}{Aufbau und physische Abgrenzung des TOE}

Der TOE besteht aus folgenden Subsystemen:

\begin{description}
\item[\tds{sub.vpn}] enthält die VPN-Funktionen wie IPSec und IKE.
\item[\tds{sub.ntpclient}] synchronisiert die Systemzeit mit einem Zeitserver.
\item[\tds{sub.selfprotect}] enthält Schutzmechanismen für den TOE.
\item[\tds{sub.adminsystem}] wird für die Administration des TOE verwendet.
\item[\tds{sub.cryptsystem}] bietet kryptografische Basisfunktionen.
\item[\tds{sub.tls}] erstellt TLS-Verbindungen für die Administrationsschnittstelle.
\end{description}



\hrefsubsection{intro.desc.secservices}{Logische Abgrenzung: Vom TOE
  erbrachte Sicherheitsdienste}

\hrefsubsection{intro.desc.scope}{Physischer Umfang des TOE}

Der physische Umfang des TOE umfasst die in \tableref{tab:intro.desc.scope} aufgelisteten Komponenten.


\renewcommand{\arraystretch}{1.5}
\begin{table}[htb]
  \centering
  \begin{tabularx}{0.95\linewidth}{@{}p{7cm}Xl@{}}
    \toprule
    Komponente & Beschreibung & Version\\ \midrule
    Firmware Image & Die Firmware  des TOE & \toeversion{} \\
    Guidance Documentation („Administrationshandbuch“) & Die Guidance Documentation beschreibt die sichere Verwendung des TOE & \toeversion{} \\
    Benutzerhandbuch ("`Allgemeine Gebrauchsanleitung \thisproduct{}"') & Das Benutzerhandbuch beschreibt die allgemeine Verwendung, sowohl dessen TOE Anteile als auch die nicht-TOE Anteile & \toeversion\\
    \bottomrule
  \end{tabularx}
  \caption{Physischer Umfang des TOE}
  \label{tab:intro.desc.scope}
\end{table}
\renewcommand{\arraystretch}{1.0}

%%% Local Variables:
%%% mode: latex
%%% TeX-master: "../ase"
%%% End:
