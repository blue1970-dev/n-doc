\newcommand{\documenttitle}{Security Target}

\newcommand{\citepp}{\autocite{\thispp}}

\newcommand{\stformat}[1]{\textcolor{blue}{#1}}
\newcommand{\ppformat}[1]{#1}
\newcommand{\assigned}[1]{\underLine{#1}}
\newcommand{\selected}[1]{\underLine{\textit{#1}}}
\newcommand{\refined}[1]{\textbf{#1}}
\newcommand{\deleted}[1]{\refined{\strikeThrough{#1}}}

\makeatletter
\newcommand{\isempty}[2]{%
  \@ifmtarg{#1}{#2}{NO}}
\newcommand{\isnotempty}[2]{%
  \@ifnotmtarg{#1}{#2}}
\makeatother

\newcommand{\strefined}[2][]{\stformat{\refined{#2}}\isnotempty{#1}{\footnote{Refinement: \emph{#1}}}}
\newcommand{\stassigned}[2][]{\stformat{\assigned{#2}}\isnotempty{#1}{\footnote{Assignment: \emph{#1}}}}
\newcommand{\stselected}[2][]{\stformat{\selected{#2}}\isnotempty{#1}{\footnote{Selection: \emph{#1}}}}
\newcommand{\stdeleted}[2][]{\stformat{\deleted{#2}}\isnotempty{#1}{\footnote{Deletion: \emph{#1}}}}
\newcommand{\pprefined}[1]{\refined{#1}}
\newcommand{\ppdeleted}[1]{\deleted{#1}}
\newcommand{\ppassigned}[1]{\assigned{#1}}
\newcommand{\ppselected}[1]{\selected{\textit{#1}}}

\renewcommand{\sfrlink}[2][]{%
  \index{\sfrmain{#2}@\sfrindex{#2}}%
  \hyperlink{\sfrmain{#2}}{\sfr[#1]{#2}}}

\newcommand{\sfrlinknoindex}[2][]{%
  \hyperlink{\directlua{removeSfrSubComponent("#2")}}{\sfr[#1]{#2}}}

\newcommand{\objsource}[1]{\directlua{getObjectiveSource("#1")}}

\newcommand{\sfrsubsection}[1]{\index{\sfrplain{#1}@\sfr{#1}|textbf}%
  \label{\directlua{toLower("#1")}}%
  \hypertarget{\directlua{toLower("#1")}}%
  {\minisec{{\normalsize\sfrplain{#1}}\nopagebreak\\{\small\sfrtext{#1}}}%
    \vspace{1ex}}}

\newcommand{\objectivesubsection}[1]{\hypertarget{\directlua{toLower("#1")}}%
  {\minisec{{\normalsize\obj{#1}} {\small{(\objtext{#1})}}}%
    \label{\directlua{toLower("#1")}}}}

\newcommand{\spdsubsection}[1]{\hypertarget{\directlua{toLower("#1")}}%
  {\minisec{{\normalsize\spd{#1}} {\small{(\spdtext{#1})}}}%
    \label{\directlua{toLower("#1")}}}}


\newlist{sfrdefinitionlist}{description}{1}
\setlist[sfrdefinitionlist]{leftmargin=5cm,labelwidth=5cm, labelsep=!}

\newcommand{\sfrdefinition}[2]{%
  \begin{sfrdefinitionlist}
  \item[{\normalfont \hypertarget{#1}{{\sfr{#1}}}}] #2
  \end{sfrdefinitionlist}}



\DeclareNewTOC[listname={Verzeichnis der ST-Anwendungshinweise}]{appnote}

\newcounter{appnotenumber}

\newcommand{\applicationnote}[3][]{%
  \refstepcounter{appnotenumber}
  \begin{sfrdefinitionlist}
  \item[{\normalfont ST-Anwendungshinweis \arabic{appnotenumber}}] #3
  \end{sfrdefinitionlist}
  \addcontentsline{appnote}{section}{\arabic{appnotenumber}\quad{}#2 \isnotempty{#1}{#1}}}

\newcommand{\appnoteref}[1]{ST-Anwendungshinweis~\ref*{#1}}

\newcommand{\hierarchicalto}[1]{%
  \begin{sfrdefinitionlist}
  \item[{\normalfont Hierarchical to:}] #1
  \end{sfrdefinitionlist}}

\newcommand{\dependencies}[1]{%
  \begin{sfrdefinitionlist}
  \item[{\normalfont Dependencies:}] #1
  \end{sfrdefinitionlist}}

\newcommand{\refinement}[1]{%
  \begin{sfrdefinitionlist}
  \item[{\normalfont Refinement:}] #1
  \end{sfrdefinitionlist}}


\newlist{sfrenumeration}{enumerate}{1}
\setlist[sfrenumeration]{leftmargin=1cm,labelindent=0cm,labelwidth=1.5em,labelsep=!}
\setlist[sfrenumeration,1]{label=(\arabic*)}
%\setlist[sfrenumeration,2]{label=(\alph*)} Wie einen eigenen Zähler für die zweite Ebene einführen?

\newlist{sfritemize}{itemize}{1}
\setlist[sfritemize,1]{labelindent=0cm,labelwidth=1.5em,labelsep=!,label=\textbullet}
%labelsep=0.5em,labelwidth=1em,itemsep=0ex

\newcommand{\appnote}{\textsl{Application note}}

\newcommand{\objunmodifiedfrompp}[1]{Das in Abschnitt~\objsource{#1} von
  \citepp{} beschriebene Sicherheitsziel \obj{#1} muss erfüllt werden.}

\newcommand{\ospunmodifiedfrompp}[1]{Die in Abschnitt~\spdsource{#1} von
  \citepp{} beschriebene organisatorische Sicherheitspolitik \spd{#1} gilt ohne
  Anpassung.}

\newcommand{\threatunmodifiedfrompp}[1]{Die in Abschnitt~\spdsource{#1} von
  \citepp{} beschriebene Bedrohung \spd{#1} gilt ohne
  Anpassung.}

\newcommand{\assumptionunmodifiedfrompp}[1]{Die in Abschnitt~\spdsource{#1} von
  \citepp{} beschriebene Annahme \spd{#1} gilt ohne
  Anpassung.}


\newcommand{\sfrunmodifiedfrompp}[1]{Die in \autocite[Abschnitt~#1]{\thispp}
  formulierten Sicherheitsanforderungen gelten ohne Anpassung.}

\newcommand{\unmodifiedfrompp}{Die Verfeinerungen in \citepp{} gelten ohne Anpassung.}

\newcommand{\operationrequired}[1]{\todo[Operation required]{{[}#1{]}}}

\newcolumntype{O}{p{2.3mm}}
\newcolumntype{T}{p{2.7mm}}

\newcommand{\tracehead}[1]{\rot{\hyperlink{\directlua{toLower("#1")}}%
    {\textsmaller[2]{\textsf{\objplain{#1}}}}}}

\newcounter{compreq}
\newcommand{\thecompreqid}{COMP-REQ-\arabic{compreq}}
\newcommand{\compreqid}[1]{\refstepcounter{compreq}\secitemformat{\thecompreqid}\label{#1}}
\newcommand{\compreq}[3]{%
%  \refstepcounter{compreq}%
  \minisec{\compreqid{#3} #1}
  #2
}

%%% Local Variables:
%%% mode: latex
%%% TeX-master: shared
%%% TeX-engine: luatex
%%% End:
