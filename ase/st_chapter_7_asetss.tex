\hrefchapter{sf}{TOE Summary Specification}

This chapter provides an overview of the TOE's IT security functionalities as
described in the Functional Specification. It contains descriptions of the
general technical processes the TOE uses to fulfill the security requirements.

Section \hyperref[sf.relationtosfr]{\ref*{sf.relationtosfr}} shows the relation
between the security functionalities of the TOE and the SFR in a tabular form.
 
\hrefsection{sf.vpn}{\secfuncheadline{sf.vpn}}

The security function \secfunc{sf.vpn} provides secure communication channels
between the TOE and a remote trusted IT product.

\implementedsfr{\sfrlink{ftp_itc.1/vpn} & \sfrlink{fcs_ckm.2/ike}}

Certificates are mathematically checked.

\implementedsfr{\sfrlink{fpt_tdc.1/zert}}

\hrefsection{sf.networkservices}{\secfuncheadline{sf.networkservices}}

The security function \secfunc{sf.networkservices} provides reliable time stamps
to the TOE. A reference time is received from a trusted NTP server via NTP in
version~4 \cite{rfc5905}. The deviation between network time and local time must
not exceed 10 minutes. If the deviation is bigger, the TOE will terminate the
connection. The TOE uses time stamps to check the validity of certificates.

\implementedsfr{\sfrlink{fpt_stm.1}}

\hrefsection{sf.selfprotection}{\secfuncheadline{sf.selfprotection}}

The security function \secfunc{sf.selfprotection} is responsible to protect the
TOE and its data from manipulation.

Sensitive data are deleted from memory as soon as they are not used
anymore. This comprises cryptographic keys, ephemeral keys, session keys, and
sensitive user data. Deletion is conducted overwriting memory areas with
constant values.

\implementedsfr{\sfrlink{fdp_rip.1}}

The TOE can run a number of self tests to verify its integrity and and the
functionality of itself and its components. Tests are run automatically at boot
up. The administrator can run the tests at any time.

\implementedsfr{\sfrlink{fpt_tst.1}}

\hrefsection{sf.administration}{\secfuncheadline{sf.administration}}

The security functions define the role \keyword{Administrator}. Users with the
role \keyword{Administrator} access the TOE via a mutually authenticated TLS
connection. This connection is provided by
\secfunclink{sf.cryptographicservices} bereit gestellt. Upon authentication, administrators are authorized to configure TSF parameters and run TSF-related operations:

\begin{itemize}
\item Modify the clock
\item Run self-tests (cf. \secfunclink{sf.selfprotection})
\end{itemize}

The TOE informs the administrator about critical operational states with the
LEDs on the front of the case (\intf{PS.LED}).

\implementedsfr{\sfrlink{ftp_trp.1/admin}}

\hrefsection{sf.cryptographicservices}{\secfuncheadline{sf.cryptographicservices}}

The security function \secfunc{sf.cryptographicservices} provides
implementations of cryptographic algorithms that can be used by other TSF.

\minisec{Key Management}

The SF provides algorithms for creation, distribution and destruction of
cryptographic keys.  und Vernichtung von Schlüsseln zur Verfügung.

\implementedsfr{\sfrlink{fcs_ckm.1} & \sfrlink{fcs_ckm.4}}

\minisec{Random Numbers}

The TOE contains a DRNG according to \sfrlink{fcs_rng.1/hashdrbg} to create
random numbers of high quality. The random numbers generated by the SF are used
for TLS establishing connections (\sfrlink{fcs_ckm.1} and
\sfrlink{fcs_cop.1/tls.aes}).

\implementedsfr{\sfrlink{fcs_rng.1/hashdrbg}}

\minisec{Hash-Algorithms}

The SF provides implementations for hash algorithms SHA-256 and SHA-512.

\implementedsfr{\sfrlink{fcs_cop.1/hash}}

\minisec{HMAC Generation}

The SF provides algorithms for HMAC generation with HMAC-SHA-256(-128).

\implementedsfr{\sfrlink{fcs_cop.1/hmac}}

\hrefsection{sf.tls}{\secfuncheadline{sf.tls}}

The TOE provides an implementation of the TLS protocol in version~1.2. The SF
ensures the integrity and confidentiality of connections to the administrator's
web browser. Table~\vref{tab:tlsconnections} lists all parameters and connections.

\implementedsfr{\sfrlink{ftp_itc.1/tls} & \sfrlink{fcs_cop.1/tls.aes} & \sfrlink{fcs_ckm.2/tls}}

The SF \secfunc{sf.cryptographicservices} provides algorithms for signature
verification. X.509 certificates are validated with RSA-PKCS1-v1.5 or
RSASSA-PSS.

\implementedsfr{\sfrlink{fpt_tdc.1/tls.zert} & \sfrlink{fcs_cop.1/tls.auth}}


\pagebreak

\hrefsection{sf.relationtosfr}{Relation of SFR to SF}

\tableref{tab:sf.mapping} shows the relation between the SFR defined in
\sectref{sfr.def} and the SF defined in this chapter.

\begin{longtable}[c]{@{}l>{\centering}*{8}{c}<{\centering}@{}}
  \toprule
  & \rot{\textsmaller[1]{\secfunclink{sf.vpn}}}
  & \rot{\textsmaller[1]{\secfunclink{sf.networkservices}}}
  & \rot{\textsmaller[1]{\secfunclink{sf.selfprotection}}}
  & \rot{\textsmaller[1]{\secfunclink{sf.administration}}}
  & \rot{\textsmaller[1]{\secfunclink{sf.cryptographicservices}}}
  & \rot{\textsmaller[1]{\secfunclink{sf.tls}}}
  \\
  % 
  \midrule \endhead
  \midrule \caption{Abbildung der SFR auf Sicherheitsfunktionalität}  \endfoot
  \bottomrule \caption{Abbildung der SFR auf Sicherheitsfunktionalität} \label{tab:sf.mapping} \endlastfoot
  %                                                  & vpn     & network & selfpro & adminis & cryptsrv  & tls
\textsmaller[1]{\sfrlinknoindex{ftp_itc.1/vpn}}      & \tcheck & \tno    & \tno    & \tno    & \tno      & \tno      \\
\textsmaller[1]{\sfrlinknoindex{fpt_tdc.1/zert}}     & \tcheck & \tno    & \tno    & \tno    & \tno      & \tno      \\
\textsmaller[1]{\sfrlinknoindex{fpt_stm.1}}          & \tno    & \tcheck & \tno    & \tno    & \tno      & \tno      \\
\textsmaller[1]{\sfrlinknoindex{fdp_rip.1}}          & \tno    & \tno    & \tcheck & \tno    & \tno      & \tno      \\
\textsmaller[1]{\sfrlinknoindex{fpt_tst.1}}          & \tno    & \tno    & \tcheck & \tno    & \tno      & \tno      \\
\textsmaller[1]{\sfrlinknoindex{ftp_trp.1/admin}}    & \tno    & \tno    & \tcheck & \tcheck & \tno      & \tno      \\
\textsmaller[1]{\sfrlinknoindex{fcs_cop.1/hash}}     & \tno    & \tno    & \tno    & \tno    & \tcheck   & \tno      \\
\textsmaller[1]{\sfrlinknoindex{fcs_cop.1/hmac}}     & \tno    & \tno    & \tno    & \tno    & \tcheck   & \tno      \\
\textsmaller[1]{\sfrlinknoindex{fcs_ckm.1}}          & \tno    & \tno    & \tno    & \tno    & \tcheck   & \tno      \\
\textsmaller[1]{\sfrlinknoindex{fcs_ckm.2/ike}}      & \tcheck & \tno    & \tno    & \tno    & \tno      & \tno      \\
\textsmaller[1]{\sfrlinknoindex{fcs_ckm.2/tls}}      & \tno    & \tno    & \tno    & \tno    & \tno      & \tcheck   \\
\textsmaller[1]{\sfrlinknoindex{fcs_ckm.4}}          & \tno    & \tno    & \tno    & \tno    & \tcheck   & \tno      \\
\textsmaller[1]{\sfrlinknoindex{fcs_rng.1/hashdrbg}} & \tno    & \tno    & \tno    & \tno    & \tadded   & \tno      \\
\textsmaller[1]{\sfrlinknoindex{ftp_itc.1/tls}}      & \tno    & \tno    & \tno    & \tno    & \tno      & \tcheck   \\
\textsmaller[1]{\sfrlinknoindex{fpt_tdc.1/tls.zert}} & \tno    & \tno    & \tno    & \tno    & \tno      & \tcheck   \\
\textsmaller[1]{\sfrlinknoindex{fcs_cop.1/tls.aes}}  & \tno    & \tno    & \tno    & \tno    & \tno      & \tcheck   \\
\textsmaller[1]{\sfrlinknoindex{fcs_cop.1/tls.auth}} & \tno    & \tno    & \tno    & \tno    & \tno      & \tcheck   \\

\end{longtable}

% !TEX root = ../ase
%%% Local Variables:
%%% mode: latex
%%% TeX-master: "../ase"
%%% TeX-engine: luatex
%%% End:



% !TEX root = ase
%%% Local Variables:
%%% mode: latex
%%% TeX-master: "ase"
%%% TeX-engine: luatex
%%% End:
