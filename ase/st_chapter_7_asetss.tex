\hrefchapter{sf}{TOE Summary Specification}

Dieses Kapitel vermittelt einen Überblick über die IT-Sicherheitsfunktionen des
TOE, wie sie in der funktionalen Spezifikation beschrieben sind. Es enthält
Beschreibungen der allgemeinen technischen Verfahren, die der TOE anwendet, um
die Sicherheitsanforderungen zu erfüllen.

Der Abschnitt \hyperref[sf.relationtosfr]{\ref*{sf.relationtosfr}} zeigt
tabellarisch die Zusammenhänge zwischen den Sicherheitsfunktionen des TOE und
den Sicherheitsanforderungen, die dieses Security Target in den Abschnitten
\hyperref[sfr.def]{\ref*{sfr.def}} aufstellt.
 


\hrefsection{sf.vpn}{\secfuncheadline{sf.vpn}}

Die Sicherheitsfunktion \secfunc{sf.vpn} erstellt sichere Kommunikationskanäle
zwischen dem TOE und einem entfernten, vertrauenswürdigen IT-Produkt.

\umgesetztesfr{\sfrlink{ftp_itc.1/vpn} & \sfrlink{fcs_ckm.2/ike}}

Zertifikate werden mathematisch geprüft

\umgesetztesfr{\sfrlink{fpt_tdc.1/zert}}

\hrefsection{sf.networkservices}{\secfuncheadline{sf.networkservices}}

Die Sicherheitsfunktion \secfunc{sf.networkservices} stellt dem TOE zuverlässige
Zeitstempel zur Verfügung. Eine Referenzzeit wird über den VPN-Kanal von einem
vertrauenswürdigen NTP-Server bezogen. Dabei wird NTP in Version~4 verwendet
\cite{rfc5905}. Die Abweichung zwischen der Netzwerkzeit und der lokalen Zeit im
TOE darf maximal 1~Stunde betragen. Der TOE verwendet die Uhrzeit hauptsächlich,
um die Gültigkeit von Zertifikaten zu prüfen.

\umgesetztesfr{\sfrlink{fpt_stm.1}}

\hrefsection{sf.selfprotection}{\secfuncheadline{sf.selfprotection}}

Die Sicherheitsfunktion \secfunc{sf.selfprotection} ist dafür
verantwortlich, den TOE und die Daten, die er verarbeitet, vor
Angriffen und Manipulation zu schützen.

Sensible Daten werden aus dem Arbeitsspeicher gelöscht, sobald sie
nicht mehr verwendet werden. Das umfasst kryptographische Schlüssel,
Session Keys, kurzlebige Schlüssel während des Ver- und
Entschlüsselungsvorgangs, aber auch sensible Benutzerddaten. Das
Löschen wird durch aktives Überschreiben der entsprechenden
Speicherbereiche mit einer Konstante oder pseudo-zufälligen Werten
umgesetzt.

\umgesetztesfr{\sfrlink{fdp_rip.1}}

Der TOE kann eine Reihe von Selbsttests ausführen, um seine
Integrität und die Funktionsfähigkeit seiner eigenen
Sicherheitsfunktionen und Komponenten zu beweisen. Abhängig von deren
Ausprägung werden die Selbsttests entweder beim Systemstart, während
des normalen Betriebs oder zu beiden Gelegenheiten ausgeführt. Der
Administrator kann die Selbsttests ebenfalls starten.

\umgesetztesfr{\sfrlink{fpt_tst.1}}


\hrefsection{sf.administration}{\secfuncheadline{sf.administration}}

Die Sicherheitsfunktionen des TOE definieren eine Rolle „Administrator“.
Benutzer greifen zur Verwaltung des TOE über eine TLS-Verbindung auf den TOE zu.
Die TLS-Verbindung wird von der Funktion
\secfunclink{sf.cryptographicservices} bereit gestellt. Ist ein Administrator
authentisiert, ist er autorisiert, verschiedene TSF-Parameter zu konfigurieren
und TSF-bezogene Operationen durchzuführen.

\begin{itemize}
\item Die Systemzeit/Echtzeituhr modifizieren
\item Die Selbsttests des TOE auslösen (vgl. \secfunclink{sf.selfprotection})
\end{itemize}

Es ist zu beachten, dass die Web-Anwendung in der Umgebung des TOE
ausgeführt wird. Die Sicherheitsleistungen werden von der
Management-Schnittstelle erbracht, die den Authentisierungsstatus des
Administrators prüft.

Der TOE informiert den Administrator über kritische Betriebszustände über die
LEDs an der Gehäusefront (\intf{PS.LED}).

\umgesetztesfr{\sfrlink{ftp_trp.1/admin}}

\hrefsection{sf.cryptographicservices}{\secfuncheadline{sf.cryptographicservices}}

Die Sicherheitsfunktion \secfunc{sf.cryptographicservices} stellt
Implementierungen verschiedener kryptographischer Basisalgorithmen zur
Verfügung, die von anderen Sicherheitsfunktionen des TOE verwendet werden
können.

\minisec{Schlüsselbehandlung}

Die Sicherheitsfunktionalität stellt alle Algorithmen zur Erzeugung, Verteilung
und Vernichtung von Schlüsseln zur Verfügung.

\umgesetztesfr{\sfrlink{fcs_ckm.1} & \sfrlink{fcs_ckm.4}}

\minisec{Zufallszahlen}

Der TOE enthält einen DRNG nach \sfrlink{fcs_rng.1/hashdrbg}, um Zufallszahlen
hoher Qualität zu erzeugen. Die so erzeugten Zufallszahlen werden für
verschiedene Zwecke verwendet, u.a. beim TLS-Verbindungsaufbau
(\sfrlink{fcs_ckm.1} und \sfrlink{fcs_cop.1/tls.aes})

\umgesetztesfr{\sfrlink{fcs_rng.1/hashdrbg}}

\minisec{Hash-Algorithmen}

Die Funktion bietet Implementierungen für die Hash-Algorithmen SHA-1,
SHA-256 und SHA-512. Im Kontext von TLS implementiert der TOE außerdem
SHA-384 für bestimmte Cipher Suites.

\umgesetztesfr{\sfrlink{fcs_cop.1/hash}}

\minisec{HMAC Generierung}

Die Funktion bietet darüber hinaus Algorithmen für die HMAC-Generierung, wobei
die genannten Hash-Algorithmen zum Tragen kommen: HMAC-SHA-1(-96),
HMAC-SHA-256(-128).

\umgesetztesfr{\sfrlink{fcs_cop.1/hmac}}

\hrefsection{sf.tls}{\secfuncheadline{sf.tls}}

Der TOE stellt die Umsetzung des TLS-Protokolls in der Version~1.2 bereit. Die
Funktion stellt die Integrität und Vertraulichkeit der Verbindungen zum
Web-Browswer des Administrators sicher. Die genaue Verwendung der
TLS-Verbindungen und eine Auflistung der Kommunikationspartner befindet sich in
Tabelle~\vref{tab:tlsconnections}.

\umgesetztesfr{\sfrlink{ftp_itc.1/tls} & \sfrlink{fcs_cop.1/tls.aes} & \sfrlink{fcs_ckm.2/tls}}

Die Sicherheitsfunktion \secfunc{sf.cryptographicservices} bietet Algorithmen
zur Verifikation von Signaturen. X.509-Zertifikate werden unter Verwendung des
RSA-PKCS1-v1.5- bzw RSASSA-PSS-Algorithmus geprüft.

\umgesetztesfr{\sfrlink{fpt_tdc.1/tls.zert} & \sfrlink{fcs_cop.1/tls.auth}}

Für die Generierung von Nonces und Schlüsseln verwendet der TOE den Hash\_DRBG
Zufallsgenerator aus \secfunclink{sf.cryptographicservices}. Session Keys werden
durch das Überschreiben mit konstanten oder pseudozufälligen Werten sicher aus
dem Speicher entfernt, ebenfalls durch Aufruf von
\secfunclink{sf.cryptographicservices}.


\pagebreak

\hrefsection{sf.relationtosfr}{Verhältnis von SFR zu SF}

\tableref{tab:sf.mapping} zeigt, in welchem Verhältnis die im
Abschnitt~\sectref{sfr.def} definierten Sicherheitsanforderungen an den
TOE zu den in \sectref{sf} beschriebenen Sicherheitsfunktionen des NK
stehen. Die verwendeten Symbole sind in der Legende in
\tableref{tab:o.mappinglegende} beschrieben.

\begin{longtable}[c]{@{}l>{\centering}*{8}{c}<{\centering}@{}}
  \toprule
  & \rot{\textsmaller[1]{\secfunclink{sf.vpn}}}
  & \rot{\textsmaller[1]{\secfunclink{sf.networkservices}}}
  & \rot{\textsmaller[1]{\secfunclink{sf.selfprotection}}}
  & \rot{\textsmaller[1]{\secfunclink{sf.administration}}}
  & \rot{\textsmaller[1]{\secfunclink{sf.cryptographicservices}}}
  & \rot{\textsmaller[1]{\secfunclink{sf.tls}}}
  \\
  % 
  \midrule \endhead
  \midrule \caption{Abbildung der SFR auf Sicherheitsfunktionalität}  \endfoot
  \bottomrule \caption{Abbildung der SFR auf Sicherheitsfunktionalität} \label{tab:sf.mapping} \endlastfoot
  %                                                  & vpn     & network & selfpro & adminis & cryptsrv  & tls
\textsmaller[1]{\sfrlinknoindex{ftp_itc.1/vpn}}      & \tcheck & \tno    & \tno    & \tno    & \tno      & \tno      \\
\textsmaller[1]{\sfrlinknoindex{fpt_tdc.1/zert}}     & \tcheck & \tno    & \tno    & \tno    & \tno      & \tno      \\
\textsmaller[1]{\sfrlinknoindex{fpt_stm.1}}          & \tno    & \tcheck & \tno    & \tno    & \tno      & \tno      \\
\textsmaller[1]{\sfrlinknoindex{fdp_rip.1}}          & \tno    & \tno    & \tcheck & \tno    & \tno      & \tno      \\
\textsmaller[1]{\sfrlinknoindex{fpt_tst.1}}          & \tno    & \tno    & \tcheck & \tno    & \tno      & \tno      \\
\textsmaller[1]{\sfrlinknoindex{ftp_trp.1/admin}}    & \tno    & \tno    & \tcheck & \tcheck & \tno      & \tno      \\
\textsmaller[1]{\sfrlinknoindex{fcs_cop.1/hash}}     & \tno    & \tno    & \tno    & \tno    & \tcheck   & \tno      \\
\textsmaller[1]{\sfrlinknoindex{fcs_cop.1/hmac}}     & \tno    & \tno    & \tno    & \tno    & \tcheck   & \tno      \\
\textsmaller[1]{\sfrlinknoindex{fcs_ckm.1}}          & \tno    & \tno    & \tno    & \tno    & \tcheck   & \tno      \\
\textsmaller[1]{\sfrlinknoindex{fcs_ckm.2/ike}}      & \tcheck & \tno    & \tno    & \tno    & \tno      & \tno      \\
\textsmaller[1]{\sfrlinknoindex{fcs_ckm.2/tls}}      & \tno    & \tno    & \tno    & \tno    & \tno      & \tcheck   \\
\textsmaller[1]{\sfrlinknoindex{fcs_ckm.4}}          & \tno    & \tno    & \tno    & \tno    & \tcheck   & \tno      \\
\textsmaller[1]{\sfrlinknoindex{fcs_rng.1/hashdrbg}} & \tno    & \tno    & \tno    & \tno    & \tadded   & \tno      \\
\textsmaller[1]{\sfrlinknoindex{ftp_itc.1/tls}}      & \tno    & \tno    & \tno    & \tno    & \tno      & \tcheck   \\
\textsmaller[1]{\sfrlinknoindex{fpt_tdc.1/tls.zert}} & \tno    & \tno    & \tno    & \tno    & \tno      & \tcheck   \\
\textsmaller[1]{\sfrlinknoindex{fcs_cop.1/tls.aes}}  & \tno    & \tno    & \tno    & \tno    & \tno      & \tcheck   \\
\textsmaller[1]{\sfrlinknoindex{fcs_cop.1/tls.auth}} & \tno    & \tno    & \tno    & \tno    & \tno      & \tcheck   \\

\end{longtable}

% !TEX root = ../ase
%%% Local Variables:
%%% mode: latex
%%% TeX-master: "../ase"
%%% TeX-engine: luatex
%%% End:



% !TEX root = ase
%%% Local Variables:
%%% mode: latex
%%% TeX-master: "ase"
%%% TeX-engine: luatex
%%% End:
