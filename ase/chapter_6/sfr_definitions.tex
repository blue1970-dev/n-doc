\hrefsection{sfr.def}{Funktionale Sicherheitsanforderungen}

\hrefsubsection{sfr.def.vpnclient}{VPN Client}

\sfrsubsection{ftp_itc.1/vpn}
\sfrdefinition{ftp_itc.1.1/vpn}{\sfrunmodifiedfrompp{6.2.1}}
\sfrdefinition{ftp_itc.1.2/vpn}{\sfrunmodifiedfrompp{6.2.1}}
\sfrdefinition{ftp_itc.1.3/vpn}{\sfrunmodifiedfrompp{6.2.1}}

\hrefsubsection{sfr.def.net}{Netzdienste}

\sfrsubsection{fpt_stm.1}

\sfrdefinition{fpt_stm.1.1}{\sfrunmodifiedfrompp{6.2.3}}

\refinement{Die Zuverlässigkeit (reliable) des Zeitstempels wird durch
  Zeitsynchronisation der Echtzeituhr (gemäß \objlink{oe.echtzeituhr}) mit
  Zeitservern unter Verwendung des Protokolls NTPv4 \cite{rfc5905} erreicht.
  Der EVG verwendet den verlässlichen Zeitstempel für sich selbst.}


\sfrsubsection{fpt_tdc.1/zert}
\sfrdefinition{fpt_tdc.1.1/zert}{\sfrunmodifiedfrompp{6.2.3}}
\sfrdefinition{fpt_tdc.1.2/zert}{The TSF shall use \ppselected{interpretation
    rules} when interpreting the TSF data from another trusted IT
  product.\par{}\strefined{The interpretation rules are defined in \dots{}}}

\hrefsubsection{sfr.def.packetinspection}{Stateful Packet Inspection}

\intentionallyleftblank{}

\hrefsubsection{sfr.def.selfprot}{Selbstschutz}

\sfrsubsection{fdp_rip.1}

\sfrdefinition{fdp_rip.1.1}{The TSF shall ensure that any previous information
  content of a resource is made unavailable upon the \ppselected{deallocation of
    the resource from} the following objects: \ppassigned{cryptographic keys
    (and session keys) used for the VPN or for TLS-connections},
  \stassigned[list of objects]{no other objects}.}

\refinement{Die sensitive Daten müssen mit konstanten oder zufälligen Werten
  überschrieben werden, sobald sie nicht mehr verwendet werden.\par
  \strefined{These sensitive objects are overwritten with constant or
    pseudo-random values.}}

\sfrsubsection{fpt_tst.1}

\sfrdefinition{fpt_tst.1.1}{The TSF shall run a suite of self tests
  \stselected[during initial start-up, periodically during normal
  operation, at the request of the authorised user, at the conditions
  {[}assignment: conditions under which self test should
  occur{]}]{during initial start-up, at the request of the authorised
    user} to demonstrate the correct operation of
  \stselected[{[}assignment: parts of TSF{]}, the TSF]{stored TSF
    executable code}.}

\sfrdefinition{fpt_tst.1.2}{The TSF shall provide authorised users
  with the capability to verify the integrity of \ppassigned{TSF
    data}.}

\sfrdefinition{fpt_tst.1.3}{The TSF shall provide authorised users with the
  capability to verify the integrity of \stselected[{[}assignment: parts of
  TSF{]}, the TSF]{the TSF}.}


\sfrsubsection{ftp_trp.1/admin}

\sfrdefinition{ftp_trp.1.1/admin}{The TSF shall provide a
  communication path between itself and \stselected[remote,
  local]{local} users that is logically distinct from other
  communication paths and provides assured identification of its end
  points and protection of the communicated data from
  \stselected[modification, disclosure, {[}assignment: other types of
  integrity or confidentiality violation{]}]{modification,
    disclosure}.}

\sfrdefinition{ftp_trp.1.2/admin}{The TSF shall permit \stselected[the TSF,
  local users, remote users]{local users} to initiate communication via the
  trusted path.}

\sfrdefinition{ftp_trp.1.3/admin}{\sfrunmodifiedfrompp{6.2.6}}

\applicationnote{\sfr{ftp_trp.1/admin}}{\label{appnote:clientcert} Der TOE
  unterstützt beidseitige Authentisierung durch Zertifikate im TLS-Handshake.}


\hrefsubsection{sfr.def.crypto}{Kryptographische Basisdienste}

\sfrsubsection{fcs_cop.1/hash}

\sfrdefinition{fcs_cop.1.1/hash}{The TSF shall perform
  \ppassigned{hash value calculation} in accordance with a specified
  cryptographic algorithm \ppassigned{\stdeleted{SHA-1,} SHA-256,}
  \stassigned[list of SHA-2 Algorithms with more than 256 bit
  size]{SHA-512} and cryptographic key sizes \ppassigned{none} that
  meet the following: \ppassigned{FIPS PUB 180-4
    \autocite{FIPS180-4}}.}

\sfrsubsection{fcs_cop.1/hmac} \sfrdefinition{fcs_cop.1.1/hmac}{The TSF shall
  perform \ppassigned{HMAC value generation and verification} in accordance with
  a specified cryptographic algorithm \ppassigned{HMAC with \stdeleted{SHA-1,}}
  \stassigned[list of SHA-2 Algorithms with 256bit size or more]{SHA-256} and
  cryptographic key sizes \stassigned[cryptographic key sizes]{160 and 256 bit}
  that meet the following: \ppassigned{FIPS PUB 180-4 \autocite{FIPS180-4},
    RFC~2404 \autocite{rfc2404}, RFC~4868 \autocite{rfc4868}, RFC~5996
    \autocite{rfc5996}}.}



\sfrsubsection{fcs_ckm.1}

\sfrdefinition{fcs_ckm.1.1}{The TSF shall generate cryptographic keys in
  accordance with a specified cryptographic key generation algorithm
  \stassigned[cryptographic key generation algorithm]{PRF-HMAC-SHA256} and
  specified cryptographic key sizes \stassigned[cryptographic key
  sizes]{256~bit} that meet the following: \ppassigned{TR-03116
    \autocite{TR03116}}.\par
  \strefined{The following algorithms and preferences are supported for TLS key
    negotiation}
  \begin{sfritemize}
  \item \strefined{Diffie-Hellman Group 14 according to \rfc{3526}
      \cite{rfc3526} for key establishment during TLS}
  \item \strefined{DH exponent shall have a minimum length of 384
      bits}
  \item \strefined{Forward secrecy shall be provided}
  \item \strefined{Ephemeral elliptic curve DH key exchange supports
      the \mbox{P-256} and the \mbox{P-384} curves according to FIPS186-4
      \cite{FIPS186-4} as well as the brainpoolP256r1 and the
      brainpoolP384r1 curves according to \rfc{5639} and \rfc{7027}
      \cite{rfc5639,rfc7027}}
  \item \strefined{Peer authentication (if required): X.509
      certificate with RSA 2048 bit keys}
  \end{sfritemize}
}


\sfrsubsection{fcs_ckm.2/ike}
\sfrdefinition{fcs_ckm.2.1/ike}{\sfrunmodifiedfrompp{6.2.7}}

\sfrsubsection{fcs_ckm.2/tls}
\sfrdefinition{fcs_ckm.2.1/tls}{\sfrunmodifiedfrompp{6.2.7}}

\sfrsubsection{fcs_ckm.4}

\sfrdefinition{fcs_ckm.4.1}{The TSF shall destroy cryptographic
  keys in accordance with a specified cryptographic key destruction
  method \stassigned[cryptographic key destruction method]{by
    overwriting with zeros} that meets the following: \stassigned[list
  of standards]{none}.}


\hrefsubsection{sfr.def.cryptotls}{TLS-Kanäle unter Nutzung sicherer
  kryptographischer Algorithmen}

\sfrsubsection{ftp_itc.1/tls}
\sfrdefinition{ftp_itc.1.1/tls}{\sfrunmodifiedfrompp{6.2.8}}
\sfrdefinition{ftp_itc.1.2/tls}{\sfrunmodifiedfrompp{6.2.8}}

\sfrdefinition{ftp_itc.1.3/tls}{The TSF shall initiate
  communication via the trusted channel for \ppassigned{communication
    required by the administration interface} \stassigned[list of other
  functions for which a trusted channel is required]{any
    connection specified in Table~\ref{tab:tlsconnections}.}}

\applicationnote{\sfr{ftp_itc.1/tls}}{\label{appnote:brainpool} Der TOE
  unterstützt den ECDHE-Schlüsselaustausch mit elliptischen Kurven. Die
  verwendeten Kurven sind in \tableref{tab:elliptic-curves} aufgelistet.}


\sfrsubsection{fpt_tdc.1/tls.zert}

\sfrdefinition{fpt_tdc.1.1/tls.zert}{The TSF shall provide the
  capability to consistently interpret
  
  \begin{sfrenumeration}
  \item \ppassigned{X.509-Zertifikate für TLS-Verbindungen}
  \item \ppassigned{Sperrinformationen zu Zertifikaten für TLS-Verbindungen, die
    via OCSP erhalten werden}
  \item \stassigned[additional list of data types]{no other data types}
  \end{sfrenumeration}

  when shared between the TSF and another trusted IT product.}

 
\sfrdefinition{fpt_tdc.1.2/tls.zert}{The TSF shall use
  \ppselected{interpretation rules} when interpreting the TSF data from another
  trusted IT product.}

\sfrsubsection{fcs_cop.1/tls.aes}

\sfrdefinition{fcs_cop.1.1/tls.aes}{\sfrunmodifiedfrompp{6.2.8}}

\sfrsubsection{fcs_cop.1/tls.auth}

\sfrdefinition{fcs_cop.1.1/tls.auth}{\sfrunmodifiedfrompp{6.2.8}}


\hrefsubsection{sfr.def.additional}{Zusätzliche Sicherheitsanforderungen}

Dieser Abschnitt enthält Sicherheitsanforderungen, die zusätzlich zu denen des
Schutzprofils definiert werden. Die Anforderungen werden hier um die in Kapitel
\ref{comp.pp-fcsrng} definierte Anforderung \sfr{fcs_rng.1/hashdrbg}
erweitert. 

\sfrsubsection{fcs_rng.1/hashdrbg}

\hierarchicalto{No other components}
\dependencies{No dependencies}

\sfrdefinition{fcs_rng.1.1/hashdrbg}{The TSF shall provide a
  \stselected[physical, non-physical true, deterministic, hybrid
  physical, hybrid deterministic]{deterministic} random number
  generator that implements: \stassigned[list of security
  capabilities]{}
  \begin{sfrenumeration}
  \item[\stassigned{(1)}] \stassigned{If initialized with a random
      seed using PTRNG of class PTG.2 as random source, the internal
      state of the RNG shall have at least 100 bits min-entropy.}
  \item[\stassigned{(2)}] \stassigned{The RNG provides forward
      secrecy.}
  \item[\stassigned{(3)}] \stassigned{The RNG provides backward
      secrecy even if the current internal state is known.}
  \end{sfrenumeration}
}

\sfrdefinition{fcs_rng.1.2/hashdrbg}{The TSF shall provide random numbers that
  meet: \stassigned[a defined quality metric]{}
  \begin{sfrenumeration}
  \item[\stassigned{(1)}] \stassigned{The RNG gets initialized during
      every startup and after 2048 requests with a random seed of
      minimal 384 bits using a PTRNG of class PTG.2. The RNG generates
      output for which more than $2^{34}$ strings of bit length 128
      are mutually different with probability $w > 1- 2^{(-16)}$.}
  \item[\stassigned{(2)}] \stassigned{Statistical test suites cannot
      practically distinguish the random numbers from output sequences
      of an ideal RNG. The random numbers must pass test procedure A.}
  \end{sfrenumeration}}


% !TEX root = ase
%%% Local Variables:
%%% mode: latex
%%% TeX-master: "../ase"
%%% TeX-engine: luatex
%%% End:
