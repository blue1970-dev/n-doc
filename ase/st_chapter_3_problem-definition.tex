\hrefchapter{spd}{Definition des Sicherheitsproblems}

In diesem Abschnitt wird zunächst beschrieben, welche Werte der TOE
schützen muss, welche externen Einheiten mit ihm interagieren und
welche Objekte von Bedeutung sind. Auf dieser Basis wird danach
beschrieben, welche Bedrohungen der TOE abwehren muss, welche
organisatorischen Sicherheitspolitiken zu beachten sind und welche
Annahmen an seine Einsatzumgebung getroffen werden können.

\hrefsection{spd.assets}{Werte}

\hrefsubsection{spd.assets.assets}{Zu Schützende Werte}

Die \emph{zu schützenden Werte} -- also Ressourcen und Daten, die der TOE
schützt -- werden in \citepp{} beschrieben.  Die dort beschriebenen Werte gelten
ohne Anpassung.

\hrefsubsection{spd.assets.users}{Benutzer des TOE}

Die \emph{externen Entitäten, Subjekte und Objekte} des TOE werden in \citepp{}
beschrieben. Die \emph{Benutzer} des TOE werden in
\autocite[Abschnitt~3.1.1]{\thispp} beschrieben. Diese Beschreibung gilt ohne
Anpassung.  Die Subjekte, die im Auftrag des Benutzers agieren, werden in
\autocite[Abschnitt~6.1.2]{\thispp} modelliert. Auch diese Darstellung wird ohne
Anpassung in das Security Target übernommen.

\hrefsection{spd.threats}{Bedrohungen}

\spdsubsection{t.wan.client}

\threatunmodifiedfrompp{t.wan.client}

\spdsubsection{t.lan.admin}

\threatunmodifiedfrompp{t.lan.admin}

\spdsubsection{t.zert_prüf}

\threatunmodifiedfrompp{t.zert_prüf}

\spdsubsection{t.timesync}

\threatunmodifiedfrompp{t.timesync}


\hrefsection{spd.osp}{Organisatorische Sicherheitspolitiken}

\spdsubsection{osp.zeitdienst}

\ospunmodifiedfrompp{osp.zeitdienst}

\spdsubsection{osp.tls}

\ospunmodifiedfrompp{osp.tls}

\hrefsection{spd.assumptions}{Annahmen}

\spdsubsection{a.guidance}

\assumptionunmodifiedfrompp{a.guidance}


% !TEX root = ../ase
%%% Local Variables:
%%% mode: latex
%%% TeX-master: "../ase"
%%% TeX-engine: luatex
%%% End:
