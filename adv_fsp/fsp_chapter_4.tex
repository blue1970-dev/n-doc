\chapter{Security Functions of the TOE}\label{sf}

\hrefsection{sf.vpn}{\secfuncheadline{sf.vpn}}

\callerinterfaces{sf.vpn}
\configparams{sf.vpn.cfgparams}

Configuration of the security function is documented in the administrator guide
\autocite[Section~7.4.3.3 VPN (Virtual Private Network)]{agd_adm}.

\sfdescription{}

The TOE provides a VPN client that enables the TOE to establish secure channels
between itself and other IT products of type VPN concentrator. These channels
are separated logically from other channels and provide identification and
authentication of their end points. Integrity and confidentiality of transmitted
data is guaranteed. VPN conections are established on the interface \lswan{}.
 
\subsection{Authentication and Key Negotiation with IKE}\label{sf.vpn.ike}

Key neogtiation is implemented by the IKEv2 protocol according to
\rfc[c]{4306}. The TOE requires the peer to accept Diffie-Hellman-Group~14
according to \rfc[c]{3526}. The TOE uses a DH exponent of 384~bit. Initiator
(TOE) and responder (VPN concentrator) are authenticated mutually with
certificate authentication. Signature validation is performed by the security
function \secfunclink{sf.cryptographicservices}
(see section~\vref{sf.cryptographicservices}).

\implementedsfr{\sfr{ftp_itc.1/vpn} & \sfr{fcs_ckm.2/ike}}

\subsection{Certificate Verification}\label{sf.vpn.certverification}

Certificates of the VPN concentrators are checked for their mathematical
correctness, their expiry date and the revocation
information.

\implementedsfr{\sfr{fpt_tdc.1/zert}}.

\hrefsection{sf.networkservices}{\secfuncheadline{sf.networkservices}}

\callerinterfaces{sf.networkservices}

\configparams{sf.networkservices.cfgparams}

Configuration of the security function is documented in the administrator guide
\autocite[Section~7]{agd_adm}.

\sfdescription{}

This security function implements several protocols and services that support
secure TOE operation by providing access to network services offered by other IT
products in the WAN.

\hypertarget{sf.networkservices.ntp}{\subsection{NTP Client}\label{sf.networkservices.ntp}}

The TOE provides a client for the NTP protocol. The TOE uses this service to
synchronize the system time with NTP servers in the WAN. This is required by the
audit log which requires reliable time stamps.

\implementedsfr{\sfrlink{fpt_stm.1}}

\subsection{DHCP Client}\label{sf.networkservices.dhcpclient}

The TOE provides a DHCP client on its WAN interface \lswandhcp{} to obtain IP
addresses and routing information according to \rfc[c]{2131} und \rfc[c]{2132}.

\hrefsection{sf.selfprotection}{\secfuncheadline{sf.selfprotection}}

\callerinterfaces{sf.selfprotection}

\configparams{sf.selfprotection.cfgparams}

This security function has no configuration parameters.

\sfdescription{}

The security function is responsible for the self protection of the TOE and for
the protection of data transmitted via the TOE between IT products in the LAN
and WAN.

\subsection{Secure Memory Deallocation}\label{sf.selfprotection.rip}

Sensitive data and cryptographic keys are erased securely from the TOE's memory
as soon as they are not used anymore. Deletion is implemented by overwriting the
memory areas with null bytes. This security function has no TSFI and is only
called from within the TOE.

\implementedsfr{\sfrlink{fdp_rip.1}}

\subsection{Self-Tests}\label{sf.selfprotection.selftest}

The TOE implements self-tess that can be used to check integrity and correct
functionality of the TSF. The administrator can run the self-tests by calling
them via the management interface \lslanhttpmgmt{}.

\implementedsfr{\sfr{fpt_tst.1}}

\hrefsection{sf.administration}{\secfuncheadline{sf.administration}}%

\callerinterfaces{sf.administration}

\configparams{sf.administration.cfgparams}

Configuration of the security function is documented in the administrator guide
\autocite[Section~7]{agd_adm}.

\sfdescription{}

This security function provides an interface and processes for configuration of
the TOE. All operational parameters can be set with this function.

\hrefsection{sf.cryptographicservices}{\secfuncheadline{sf.cryptographicservices}}%

\callerinterfaces{sf.cryptographicservices}

\configparams{sf.cryptographicservices.cfgparams}

This security function has no configuration parameters.

\sfdescription{}

The security function provides cryptographic services that can be used by other TSF.

\subsection{Random Number Generation}%
\label{sf.cryptographicservices.rng}

The TOE contains a DRNG to generate random numbers of high quality
\autocite{SP80090A}. This RNG is used to generate the random numbers and nonces
used for establishing TLS and IPSec connections.

\implementedsfr{\sfrlink{fcs_rng.1/hashdrbg}}

\supportedsfr{\sfrlink{fcs_ckm.1} & \sfrlink{fcs_cop.1/tls.aes}}

\subsection{HMAC Algorithms}%
\label{sf.cryptographicservices.hmac}

The security function provides implementations of algorithms for HMAC
generation. The TOE supports HMAC-SHA-256(-128).

\implementedsfr{\sfrlink{fcs_cop.1/hmac}}


\subsection{Signature Verification}%
\label{sf.cryptographicservices.sigver}

The TOE supports signature verification. This is used during certificate
validation.

\subsubsection{Key Exchange for  IKE (Diffie-Hellman)}%
\label{sf.cryptographicservices.ipsec.keyexc}

The TOE implements the Diffie-Hellman algorithm for key exchange used by the IKE
protocol. The TOE does not reuse DH-keys and thus provides perfect forward
secrecy.

\implementedsfr{\sfrlink{fcs_ckm.2/ike}}

\subsubsection{Key destruction}%
\label{sf.cryptographicservices.ipsec.keydest}

The TOE destroys symmetric keys for IKE, ESP and TLS by overwriting the memory
areas with null bytes.

\implementedsfr{\sfrlink{fcs_ckm.4}}

\hrefsection{sf.tls}{\secfuncheadline{sf.tls}}

The TOE implements a TLS server in protocol version~1.2 according to
\rfc[c]{5246}. All TLS connections and their parameters are listed in
\appendixref{appendix.tls}.


% !TEX root = adv_fsp
%%% Local Variables:
%%% mode: latex
%%% TeX-engine: luatex
%%% TeX-master: "adv_fsp"
%%% End:
