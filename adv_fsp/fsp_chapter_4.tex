\chapter{Sicherheitsfunktionen des TOE}\label{sf}

\hrefsection{sf.vpn}{\secfuncheadline{sf.vpn}}

\callerinterfaces{sf.vpn}
\configparams{sf.vpn.cfgparams}

Die Konfiguration der Sicherheitsfunktion wird im Administratorhandbuch
beschrieben \autocite[Abschnitt~7.4.3.3 VPN (Virtual Private Network)]{agd_adm}.

\sfdescription{}

Der TOE stellt einen VPN Service (Client) bereit, mit dem der TOE sichere Kanäle
zwischen sich selbst und VPN-Konzentratoren aufbauen kann. Diese Kanäle werden
logisch von anderen Kanälen getrennt gehalten und stellen eine Identifikation
und Authentisierung der Endpunkte des Kanals sicher.  Durch den Kanal werden die
übertragenen Daten in Bezug auf ihre Integrität und Vertraulichkeit
geschützt. Der Aufbau geschieht über die Schnittstelle \lswan{}.
 
\subsection{Authentifizierung und Schlüsselaushandlung via IKE}\label{sf.vpn.ike}

Die Schlüsselvereinbarung wird mittels IKEv2-Protokoll nach \rfc[c]{4306} umgesetzt. 
Hierbei wird alleinig die Diffie-Hellman-Gruppe~14 nach \rfc[c]{3526} unterstützt. 
Der TOE verwendet für den Schlüsselaustausch einen DH-Exponent der Länge 384~Bit.

Bereits verwendete Diffie-Hellman-Schlüssel werden nicht erneut verwendet, 
um (Perfect) Forward Secrecy zu gewährleisten.

Initiator (TOE) und Responder (VPN-Konzentrator) werden im Rahmen des
IKE-Protokolls gegenseitig zertifikatsbasiert authentifiziert
(\sfr{ftp_itc.1/vpn}, \sfr{fcs_ckm.2/ike}) und handeln Schlüssel für die
Verschlüsselung und Integritätssicherung der Nutzdaten aus.  Die Signaturprüfung
mit öffentlichen Schlüsseln und die symmetrischen Algorithmen werden in Software
umgesetzt (s.~Abschnitt~\vref{sf.cryptographicservices}).

Der TOE erzeugt die entsprechenden symmetrischen Schlüssel, die notwendig 
sind, um die Daten des IKE-Protokolls und des ESP-Protokolls zu verschlüsseln 
und HMACs für die Integritätssicherung zu erzeugen.

\subsection{Gültigkeitsprüfung von Zertifikaten}\label{sf.vpn.certverification}

Die Zertifikate der Konzentratoren werden durch den TOE sowohl auf ihre
mathematische Korrektheit, auf ihre zeitliche Gültigkeit geprüft. Außerdem wird
der Sperrstatus der Zertifikate unter Verwendung von CRLs geprüft
(\sfr{FPT_TDC.1/Zert}).

\hrefsection{sf.networkservices}{\secfuncheadline{sf.networkservices}}

\callerinterfaces{sf.networkservices}

\configparams{sf.networkservices.cfgparams}

Die Konfiguration der Sicherheitsfunktion wird im Administratorhandbuch
beschrieben \autocite[Abschnitt~7.4]{agd_adm}.

\sfdescription{}

Der TOE stellt die folgenden Dienste im Netzwerk zur Verfügung (In Klammern das
entsprechende Kapitel im Administratorhandbuch \autocite{agd_adm}):

\begin{itemize}
\item NTP-Client (Abschnitt~7.4.3.4)
\end{itemize}

\hypertarget{sf.networkservices.ntp}{\subsection{NTP-Client}\label{sf.networkservices.ntp}}

Der TOE implementiert einen NTP-Client. Die Sicherheitsfunktion
\secfunc{sf.networkservices} stellt anderen Komponenten des TOE
die genaue Uhrzeit bereit (\sfrlink{FPT_STM.1}).

\subsection{DHCP-Client}\label{sf.networkservices.dhcpclient}

Der TOE bietet an WAN-Schnittstelle (\lswandhcp{}) jeweils die Möglichkeit
DHCP nach \rfc[c]{2131} und \rfc[c]{2132} zu nutzen, um IP-Adressen,
Default-Routen und DNS-Server zu beziehen. Die Konfiguration wird im
Administratorhandbuch beschrieben \autocite{agd_adm}.


\hrefsection{sf.selfprotection}{\secfuncheadline{sf.selfprotection}}

\callerinterfaces{sf.selfprotection}

\configparams{sf.selfprotection.cfgparams}

Die Sicherheitsfunktion verfügt über keine Konfigurationsparameter.

\sfdescription{}

Diese Sicherheitsfunktion ist verantwortlich für den Selbstschutz des
TOE und für den Schutz von Daten, die durch den TOE übertragen werden.

\subsection{Speicheraufbereitung}\label{sf.selfprotection.rip}

Sensible Daten wie kryptografische Schlüssel werden sicher aus dem Speicher des
TOE gelöscht sobald diese nicht mehr verwendet werden.

Die Löschung wird durch aktives Überschreiben der entsprechenden
Speicherbereiche mit Nullen erreicht (\sfrlink{FDP_RIP.1}). Die jeweiligen
Implementierungen des sicheren Löschens für die verschiedenen Module des TOE
sind in der Design-Spezifikation beschrieben \cite{adv_tds}.

Diese Funktionalität hat keinerlei öffentlich verfügbare
Schnittstelle. 
\subsection{Selbsttests}\label{sf.selfprotection.selftest}

Der TOE implementiert Selbsttests, mit denen die Integrität und die
korrekte Funktionsweise des TOE überprüft werden kann.

Zur Laufzeit kann die Durchführung der Tests durch den Administrator
über die Managementanwendung gestartet werden (\sfr{FPT_TST.1}).

\hrefsection{sf.administration}{\secfuncheadline{sf.administration}}%

\callerinterfaces{sf.administration}

\configparams{sf.administration.cfgparams}

Die Funktionalität zur Konfiguration des TOE wird im Administratorhandbuch
beschrieben \autocite[Abschnitte~7.3 bis~7.6]{agd_adm}. 

\sfdescription{}

Im TOE bieten die Dienste ihre Managementfunktionen jeweils über eine dedizierte
Schnittstelle an, die durch den Administrationsdienst angesprochen wird.

\hrefsection{sf.cryptographicservices}{\secfuncheadline{sf.cryptographicservices}}%

\callerinterfaces{sf.cryptographicservices}

\configparams{sf.cryptographicservices.cfgparams}

Die Aufgabe von \secfunc{sf.cryptographicservices} ist das
Bereitstellen von TLS-Verbindungen zur Managementanwendung.

\sfdescription{}

Der TOE stellt kryptographische Funktionen bereit, die von den anderen
Sicherheitsfunktionen verwendet werden.

\subsection{Zufallszahlen}%
\label{sf.cryptographicservices.rng}

Der TOE enthält einen DRNG nach \sfrlink{fcs_rng.1/hashdrbg}, um
Zufallszahlen hoher Qualität zu erzeugen \autocite{SP80090A}.

Der von \secfunc{sf.cryptographicservices} bereitgestellte Zufallsgenerator wird
verwendet, um die Zufallszahlen und Nonces beim TLS-Verbindungsaufbau
(\sfrlink{fcs_ckm.1} und \sfrlink{fcs_cop.1/tls.aes}) zu erzeugen. Über den
TLS-Verbindungsaufbau hinaus wird der Zufallsgenerator immer dann herangezogen,
wenn Schlüssel erzeugt werden müssen.

\subsection{HMAC-Algorithmen}%
\label{sf.cryptographicservices.hmac}

Die Funktion bietet Implementierungen der Algorithmen für die
HMAC-Generierung, wobei der Hash-Algorithmus
HMAC-SHA-256(-128) umgesetzt wird (\sfrlink{fcs_cop.1/hmac}).

\subsection{Signaturverifikation}%
\label{sf.cryptographicservices.sigver}

Der TOE unterstützt die Verifikation von Signaturen.


\subsubsection{Pseudo Random Function für die Generierung
  symmetrischer Schlüssel (PRF)}%
\label{sf.cryptographicservices.ipsec.prf}

Der TOE erzeugt pseudozufällige Zahlen für die Erzeugung von symmetrischem
Schlüssel für Verschlüsselung und Integritätssicherung im Rahmen der Protokolle
IKE (\secitem{IKE_ENCR} und \secitem{IKE_INTEG}) und ESP (\secitem{ESP_ENCR} und
\secitem{ESP_INTEG}).

Der Algorithmus PRF-HMAC-SHA-256 wird vom TOE unterstützt.

\rfc{2104} spezifiziert den HMAC-Algorithmus für beliebige Hash-Algorithmen
\cite{rfc2104}.  FIPS180-4 \cite{FIPS180-4} spezifiziert die verwendeten
Hash-Algorithmen. \cite{rfc4868} beschreibt die PRFs auf Basis der HMACs, wie
sie bei IKE eingesetzt werden (\sfrlink{fcs_ckm.1}, \sfrlink{fcs_ckm.2/ike}).


\subsubsection{Schlüsselaustausch für IKE (Diffie-Hellman)}%
\label{sf.cryptographicservices.ipsec.keyexc}

Der TOE implementiert den Diffie-Hellman-Algorithmus für den Schlüsselaustausch
im Rahmen des IKE-Protokolls IKEv2 \cite{rfc7296}.  Der TOE stellt zudem sicher,
dass Diffie-Hellman-Schlüssel nur für eine Sitzung verwendet werden und danach
neu erzeugt werden, um (Perfect) Forward Secrecy zu
gewährleisten. (\sfrlink{fcs_ckm.2/ike})

\subsubsection{Schlüsselzerstörung}%
\label{sf.cryptographicservices.ipsec.keydest}

Der TOE zerstört die symmetrischen Schlüssel für IKE und ESP und die
Diffie-Hellman-Schlüssel, die im Rahmen von IKE erzeugt werden, indem er die
verwendeten Speicherbereiche nach der Verwendung komplett mit Nullen
überschreibt. (\sfrlink{fcs_ckm.4})

\hrefsection{sf.tls}{\secfuncheadline{sf.tls}}

Der TOE implementiert das TLS-Protokoll in der Version~1.2. Dabei werden die
Cipher Suites in \tableref{tab:ciphersuites} unterstützt.  Transport Layer
Security Version~1.2 ist in \rfc{5246} spezifiziert \cite{rfc5246}. Weiterhin
relevant sind \rfc{8017} (RSA) für den Nachweis des Schlüsselbesitzes
\cite{rfc8017} sowie \rfc{3526}. Neben dem Schlüsselaustausch mit Primzahlen
nach Diffie-Hellman beherrscht der TOE auch das ECDHE-Verfahren, bei dem Punkte
auf elliptischen Kurven berechnet und ausgetauscht werden. Für dieses Verfahren
werden die Kurven secp256r1, secp384r1 \cite{rfc8422, X9.62} und
brainpoolP256r1, brainpoolP384r1 \cite{rfc7027} genutzt. Alle TLS-Parameter sind
in den Tabellen in Anhang~\ref{appendix.tls} zusammengefasst.

Im Rahmen des sicheren Löschen von geheimen Schlüsselmaterial muss die
Speicherverwaltung berücksichtig werden. Die Details zur sicheren
Schlüssellöschung sind in \autocite{adv_tds} dokumentiert.
(\sfrlink{fcs_ckm.4})



% !TEX root = adv_fsp
%%% Local Variables:
%%% mode: latex
%%% TeX-engine: luatex
%%% TeX-master: "adv_fsp"
%%% End:
