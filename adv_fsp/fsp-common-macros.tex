\newcommand{\documenttitle}{Functional Specification}

\newcommand{\tsfisectionname}[1]{\texorpdfstring%
    {\tsfi{#1}}%
    {\directlua{getTsfi("#1")}}}

\newcommand{\tsfipurpose}[1]{\hrefsubsubsection{#1.terms}{\ndoc@purposeandterms}}
\newcommand{\tsfiparameters}[1]{\hrefsubsubsection{#1.params}{\ndoc@parameters}}

% Prints mapping tables 
\newcommand{\printSfToTsfi}[1]{\directlua{print_sf_to_tsfi_table("#1")}}
\newcommand{\printTsfiToSf}[1]{\directlua{print_tsfi_to_sf_table("#1")}}


\newcommand{\aufrufschnittstellen}[1]{\minisec{\ndoc@sfinterfaces}%
  Der Aufruf der Sicherheitsfunktion \secfunc{#1} erfolgt über: \printSfToTsfi{#1}.\par{}}

\newcommand{\aufrufschnittstellenmanual}[2]{\minisec{\ndoc@sfinterfaces}%
  Der Aufruf der Sicherheitsfunktion \secfunc{#1} erfolgt über:
  #2.\footnote{Diese Zuordnung lässt sich nicht durch die gemeinsame Zuordnung
    eines SFR nachvollziehen.}\par{}}


\newcommand{\aufgerufenesf}[1]{\minisec{Über das Interface aufgerufene Sicherheitsfunktionen}%
  Sicherheitsfunktionen, die aufgerufen werden über das Interface \tsfi{#1}: \printTsfiToSf{#1}}

\newcommand{\aufgerufenesfmanual}[2]{\minisec{Über das Interface aufgerufene
    Sicherheitsfunktionen}%
  Sicherheitsfunktionen, die über das Interface \tsfi{#1} aufgerufen
  werden\footnote{Diese Zuordnung lässt sich nicht durch die gemeinsame
    Zuordnung eines SFR nachvollziehen.}: #2}


\newcommand{\konfigurationsparameter}[1]{\minisec{Konfigurationsparameter der Sicherheitsfunktion}\label{#1}}
\newcommand{\keinekonfigurationsparameter}{(Die Sicherheitsfunktion verfügt über keine Konfigurationsparameter)}

\newcommand{\sfbeschreibung}{\minisec{Beschreibung der Sicherheitsfunktion}}

\newcommand{\sameprotocol}[3]{Die Implementierung von #1 an der
  #2-Schnittstelle entspricht in allen Aspekten der Implementierung an
  der LAN-Schnittstelle, vgl. Abschnitt~\vref{#3}.}

\newenvironment{configvaluestable}%
{\begin{tabularx}{1\linewidth}{@{}lp{1.5cm}Xl@{}}%
    \toprule
    Parameter & Werte & Beschreibung & Default \\ \midrule}%
    {\bottomrule\end{tabularx}}

\newcommand{\generateSfrToTsfiTable}{
  \begin{longtable}[c]{@{}ll@{\hskip .3cm}>{\small}p{6cm}@{}}
    \toprule
    \secitem{SFR} &  \secitem{TSFI} & Verwendung \\ \endfirsthead
    \toprule \secitem{SFR} & \secitem{TSFI} & Verwendung \\ \midrule \endhead
    \multicolumn{3}{r}{\rule{0pt}{3ex}\emph{Weiter auf der nächsten Seite}} \endfoot
    \endlastfoot
    \printTsfiForSfrRows{}
    \bottomrule
    \caption{Zuordnung von SFR zu TSFI}
    \label{tab:sfr2tsfi}
  \end{longtable}
}


\newcommand{\printTsfiForSfrRows}[1]{\directlua{print_tsfi_for_sfr_rows("#1")}}


%%% Local Variables:
%%% mode: latex
%%% TeX-master: "../adv_fsp"
%%% TeX-engine: luatex
%%% End:
