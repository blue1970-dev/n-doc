\chapter{Introduction}
\label{intro}

This document contains the necessary information for the evaluation of the
security assurance component \secitem{ADV_FSP.4} for the evaluation of
\thisproduct{}. The document starts by describing the physical and logical
interfaces that concern the TSF, the security functions of the TOE. After that,
the security functions are described in detail. The documentation shows the
relationship between interfaces and security functions.


\section*{Anmerkungen zur Notation}

Tabelle~\vref{tab:intro.notc} listet die in diesem Dokument verwendeten
typographischen Auszeichnungen und ihre Verwendung auf. Oftmals ist die
Abgrenzung zwischen den in der Tabelle aufgeführten Kategorien nicht ganz
leicht: So kommt es gelegentlich vor, dass auf verschiedenen Abstraktionsebenen
dieselben Begriffe verwendet werden. Es ist in solchen Fällen nicht auf den
ersten Blick zu erkennen, auf welcher Abstraktionsebene der Begriff gerade
verwendet wird. Beispielsweise kann ein Begriff gleichzeitig ein
Konfigurationsparameter aus der Spezifikation und gleichzeitig der Name einer
Variable im Code sein. Durch eine möglichst hohe typographische Konsistenz soll
der jeweilige Kontext verdeutlicht werden. Die Hinweise in
\tableref{tab:intro.notc} sollen helfen, die Begrifflichkeiten einzuordnen und
voneinander abzugrenzen.

Module und Subsysteme werden durch doppelte Doppelpunkte in der Form
\secitemformat{Sub\-sys\-tem::Mo\-dul} notiert. Wird in einem solchen Kontext
auf eine Schnittstelle verwiesen, wird der Name der Schnittstelle durch zwei
Schrägstriche vom Namen des Moduls getrennt:
\secitemformat{Sub\-sys\-tem::Mo\-dul//Schnittstelle}.

Die ihm Text zitierten Namen von Code-Elementen -- besonders die Komponenten im
Java-Umfeld -- können zum Teil lang werden. In solchen Fällen werden die Namen
der besseren Lesbarkeit halber mit Trennstrichen versehen, die in dieser Form
nicht im Code sichtbar sind. Diese minimale Abweichung wird in Kauf genommen, um
dem Text ein harmonisches Gesamtbild zu verleihen, das nicht durch übermäßigen
Weißraum gestört wird.

\begin{table}[htb]
  \centering{}
  \begin{tabularx}{\textwidth}{@{}lX@{}}
    \toprule
    Typographische Auszeichnung & Verwendung\\
    \midrule
    \keyword{Schlüsselwörter} & \keyword{Schlüsselwörter} sind solche Begriffe, die z.\,B. direkt aus der Spezifikation entnommen sind, dies können Konfigurationsparameter und ihre Werte sein. Aber auch andere Begriffe, die im Kontext des TOE eine hervorgehobene Bedeutung haben, sind so ausgezeichnet.\\
    \code{Code-Elemente} & \code{Code-Elemente} sind solche Begriffe, die unmittelbar aus dem Source-Code des TOE entnommen sind. Dies können beispielsweise Java-Bundles, Klassen- oder Methodennamen sein, aber auch deren Parameter oder logische Strukturen in einer der verwendeten Programmiersprachen.\\
    \filename{Dateinamen} & \filename{Dateinamen} beziehen sich auf Namen oder Namensteile von Elementen im Dateisystem.\\
    \secitem{Sicherheitsbezogene Begriffe} & Begriffe, die in direktem Bezug zum Rahmenwerk der Common Criteria stehen, sind als \secitem{Sicherheitsbezogene Begriffe} gesetzt. Dies sind SFR, die Namen der SF und TSFI, aber auch die Namen der Subsysteme, Module und Schnittstellen, die den TOE konstituieren. Weiterhin sind Namen der vom TOE verwendeten Zertifikate und sonstigen Schlüsselmaterials in dieser Form gesetzt.\\
    \bottomrule
  \end{tabularx}
    \caption{Typographische Konventionen}
    \label{tab:intro.notc}
\end{table}


%%% Local Variables:
%%% mode: latex
%%% TeX-master: shared
%%% TeX-engine: luatex
%%% End:


\cleardoublepage

\chapter{Physical Interfaces}\label{tsfi.ps}

\hrefsection{tsfi.ps.lan}{\texorpdfstring{\formatintf{PS.LAN}}{PS.LAN}}

The TOE uses the interface \formatintf{PS.LAN} to communicate with other IT
products in the LAN. The interface is an ethernet interface in the form of a
RJ45 jack. The logical interface \lslan{} is provided via this
interface. Despite the TOE being a software TOE, the hardware plays an important
role for the TOE's security. For this reason, we examine the properties of the
physical interfaces provided by \thisproduct{}. Sinc ethe physical properties
are identical, the description holds for \formatintf{PS.LAN} as well as for
\formatintf{PS.WAN}.


\hrefsection{tsfi.ps.wan}{\texorpdfstring{\formatintf{PS.WAN}}{PS.WAN}}

This interface is analogous to the interface \formatintf{PS.LAN}. It is used to
communicate with other IT products in the WAN.

\hrefsection{tsfi.ps.led}{\texorpdfstring{\formatintf{PS.LED}}{PS.LED}}

The TOE's case has LEDs on the front side. The LEDs indicate the TOE's
operational status. The logical interface \lsled{} is provided via this physical
interface.

% !TEX root = "../adv_fsp"
%%% Local Variables:
%%% mode: latex
%%% TeX-engine: luatex
%%% TeX-master: "../adv_fsp"
%%% TeX-parse-self: t
%%% TeX-auto-save: t
%%% End:
